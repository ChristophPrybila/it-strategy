Die heutige Generation lebt immer mehr vom Trend der Smartphones und Tablets. \glqq Jeder Mensch\grqq in den westlichen Staaten besitzt ein Smartphone, viele mitunter auch noch ein Tablet und zu guter Letzt auch noch ein Notebook und/oder einen Stand-PC. Diese Menge an Ger�ten mit denen Informationen aus dem Internet konsumiert werden k�nnen, stellt ganz neue Herausforderungen an angebotene Dienste von Unternehmen. So ist es heutzutage ein Muss, dass die Funktionen, die auf der Desktopoberfl�che zur Verf�gung stehen auch am Tablet oder Smartphone benutzbar sind. Jedoch in einer Darstellung die an die neuen Hardwarebedingungen angepasst ist. Ein Kunde m�chte somit in der Bank und auf jedem seiner Ger�te die gleichen Funktionen wie In- und Auslands�berweisung, Kontoauszug, das Sperren etwaiger Karten und vieles Mehr zur Verf�gung haben. Diese Vielfalt von Funktionen zu jeder Zeit an jedem Ort wird als \textit{Multi-Channel Banking} bezeichnet und ist zurzeit eines der wichtigsten Themen in der Bankenbranche. ~\cite{Cortinas2010}\\

Der Begriff \textit{Self-Service Technology} ist sehr verwandt mit dem \textit{Multi-Channel Banking}. Es geht dabei darum, dem Benutzer die M�glichkeit zu bieten, die Dienste die auf herk�mmliche Weise bei einem Bediensteten der Bank in Auftrag gegeben wurden und daraufhin erledigt wurden, nun selbst zu erf�llen. Diese neue Art der \glqq Bedienung\grqq wird vor Allem dadurch m�glich, dass die Verbreitung des Internets und die Zugriffsm�glichkeiten darauf durch mobile Endger�te stetig ansteigt. Diese neue \textit{Self-Service Technology} bietet dar�ber hinaus nicht nur dem Kunden Vorteile, da er von Ort und Zeit ungebunden die gew�nschten Bankgesch�fte durchf�hren kann, sondern ist auch als Strategie f�r Unternehmen �u"serst interessant, da das Personal mit direktem Kundenkontakt dadurch reduziert werden kann, als doch der Kunde pers�nlich diese Aufgaben erledigt.
Bei all den positiven Seiten darf jedoch nicht darauf vergessen werden, dass der erfolgreiche Einsatz von \textit{Self-Service Technology} und die dazugeh�rige Akzeptanz beim Kunden nur realisierbar ist, wenn bei eingesetztem \textit{Multi-Channel Banking} die angebotenen Services in jedem Channel nutzbar sind. ~\cite{Eriksson2007} \\

Um Geldtransaktionen zwischen unterschiedlichen Partnern im Internet einfacher machen zu k�nnen wurden sogenannte \textit{Web ATMs} entwickelt. Dabei handelt es sich um ein Verfahren, das dem eines EC-Karten Terminals nachgeahmt wurde. Das Bezahlen von zum Beispiel Hotelrechnungen im Internet war immer mit gewissen M�hen verbunden. So bekam man nach der Onlinebuchung zuerst eine Reservierungsbest�tigung mit Vorbehalt. Danach musste man eine �berweisung an das Hotel t�tigen und eine �berweisungsbest�tigung an den Transaktionspartner schicken. Da es sich im Unternehmen wiederum um unterschiedliche Abteilungen zur Bearbeitung von Rechnungen/Einzahlungen und Buchungsbest�tigungen handelte, konnte bis zur endg�ltigen Zusage eines Zimmerplatzes einiges an Zeit vergehen. Durch \textit{Web ATMs} ist es nun m�glich, sofort nach der Buchung auf eine Seite der Bank weitergeleitet zu werden, wo man durch Angabe der Kontodaten die �berweisung t�tigen kann und sowohl der Kunde als auch das Unternehmen sofortige Gewissheit �ber den Erfolg der Transaktion erhalten. Somit kann auch die fixe Zusage eines Zimmers gleich nach R�ckkehr auf die Unternehmensseite erfolgen. ~\cite{Tsai2010}

