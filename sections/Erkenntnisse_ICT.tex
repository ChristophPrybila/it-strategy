Um eine eigene IT Strategie zu entwickeln ist es oft sinnvoll, sich an Modellen die schon in der Praxis angewendet werden und funktionieren ein Beispiel zu nehmen. So ist es nicht verwunderlich, dass gro"se ICT Unternehmen wie IBM, Atos, Microsoft, Google und Apple - um nur ein paar zu nennen - darauf einen gro"sen Einfluss nehmen. Diese Unternehmen sind federf"uhrend in der Entwicklung, Implementierung und Bereitstellung von neuen IT Strategien wie Cloud Services, SOA und "Ahnlichem.\\

In den letzten Jahren zeigt sich ein Trend der immer weiter richtung Cloud Computing und Software as a Service (SaaS) geht. Die Vorteile liegen auf der Hand: (1) Die Illusion von endlos zur Verf"ugung stehenden Ressourcen, (2) das Fehlen von Verpflichtungen bevor man die Cloud benutzt und (3) die M"oglichkeit f"ur ben"otigte Ressourcen (Speicher, Rechenleistung, usw.) nur bei Gebrauch zu bezahlen. Diese Flexibilit"at ist vor allem f"ur kleine Unternehmen und Start-Up Unternehmen von enormem Vorteil ~\cite{Armbrust2009}. Dadurch ist es auch nicht verwunderlich, dass die Vorreiter in Punkto IT-Strategie Atos, Microsoft und Google mit Canopy, Microsoft Azure und Google Appengine bereits jeweils ein Cloudservice f"ur die public und auch private Cloud anbieten.\\

Bei der Wahl der richtigen IT-Strategie geht es laut IBM und Atos haupts"achlich um (1) die Bereitschaft der CxOs, (2) die bessere Vereinigung von IT Ressourcen, Investitionen und Gesch"aftsstrategien, (3) eine erh"ohte Effektivit"at und Effizienz des Einsatzes von IT Ressourcen und verminderte Betriebskosten, (4) die Wahl einer richtigen Cloud Strategie, (5) die Wahl einer richtigen Green IT Strategie, (6) effizienterer Einsatz von vorhandenen Ressourcen, (7) h"oherer Nutzen von Sicherheitsma"snahmen, (8) die Einhaltung von beh"ordlichen Regeln und Bestimmungen.

