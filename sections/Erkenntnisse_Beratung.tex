Ähnlich wie bei den Erkenntnisen der Industrieanalysten spielt auch die Meinung der großen Consultingunternehmen eine wichtige Rolle in Bezug auf gut konzipierte IT-Strategien. Zu den weltweit umsatzstärksten Anbieter in dieser Branche zählen McKinsey \& Company Corporation, Accenture PLC., Deloitte Consulting und Capgemini S.A, wobei jährlich Umsätze in Milliardenbereich erzielt werden. In erster Linie ist deren Kerngeschäft ihre Beratung als Dienstleistung anderen Unternehmen anzubieten. Sie beraten das Management oder Entscheidungsträger in allen möglichen Branchen beim Prozess der Entscheidungsfindung.

Vor allem im Bereich IT Business sind sehr viele Beratungsunternehmen fokussiert. Nicht zuletzt aufgrund der immer stärker wachsenden und ständig im Wandel befindlichen technologischen Herausforderungen, sind viele Firmen gezwungen, sich an die Kompetenzen, Wissen und Erfahrungen von Consultern zu stützen. Im heutigen digitalen Zeitalter bedarf es moderner und innovativer IT Strategien um nachhaltig die Kosten zu senken, Produktivität zu steigern, Geschäftseffizienz und -effektivität zu verbessern sowie letztendlich sich erfolgreich am Markt zu etablieren. Der Einsatz von technologischen Komponenten im Einklang mit der Unternehmensphilosophie verschafft einen großen Vorteil und durch dieses Zusammenwirken ist man in der Lage die Ziele des Unternehmens schneller zu erreichen. 

Laut Accenture werden im digitalen Zeitalter Business und IT Strategie zusammen verschmelzen, so dass eine klare Abgrenzung nicht mehr möglich ist. Jeder Geschäftsprozess ohne technologischen Einfluss wird in Zukunft kaum vorstellbar sein. Neue, innovative Technologien wie beispielsweise das Cloud Computing, das stetig an Populariät gewinnt und des öfteren auch schon in vielen Geschäftsbereichen Einsatz findet, werden in naher Zukunft die Strategien vieler Unternehmen beeinflussen und prägen.\cite{AccentureConsulting}