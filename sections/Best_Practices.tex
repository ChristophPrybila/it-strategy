

Seit den 1970er-Jahren galt das Gesch�ftsmodell der Universalbank in den national abgeschotteten Bankenm�rkten der deutschsprachigen L�nder und in Frankreich als vorbildlich gegen�ber den Trennbankensystemen anglo-amerikanischer Pr�gung. Als bankbetriebswirtschaftliche Vorteile dieser Organisationsform von Banken wurde vor allem der horizontale Risikoausgleich zwischen den einzelnen Banksparten und die daraus resultierende geringere Krisenanf�lligkeit von Universalbanken gesehen. In Verfolgung der ?One-Bank-Strategie? entstanden Gro�banken, die durch ein umfassendes Angebot von Bankleistungen aus einer Hand in national abgegrenzten M�rkten ausreichend Werte schafften. Doch mit der Entwicklung zur Globalisierung der Bankm�rkte und zuletzt in der aktuellen Finanzmarktkrise zeigten sich Nachteile dieses traditionellen Branchen-Gesch�ftsmodells (Modell der integrierten Bank) hinsichtlich Wirtschaftlichkeit, Kundenkommunikation und Transparenz der Wertsch�pfung. Nachzugehen ist daher der Frage, wie neue Gesch�ftsmodelle mit schlanker Bankorganisationsform, auf Kundengruppen und Kundenbed�rfnisse fokussierten Angeboten spezialisierter Banken mit ?Multi-Bank-Strategie? den Anforderungen an hohe Wirtschaftlichkeit und an Generierung sowie Verteilung von Ertr�gen geeignet sein k�nnen, das Branchen-Gesch�ftsmodell nachhaltig weiterzuentwickeln. \cite{Bieger2011}