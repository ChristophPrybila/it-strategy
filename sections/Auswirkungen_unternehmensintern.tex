Durch ihr rasantes Wachstum und gesellschaftliche Durchdringung, werden Soziale Medien immer st"arker Teil von unternehmensinternen Prozessen. Die Relevanz von Sozialen Applikationen und Technologien ist jedoch von Unternehmen zu Unternehmen unterschiedlich und schwierig zu messen. Dies macht es oftmals schwer zu entscheiden den Trend zu folgen oder auszulassen.
\cite{Nair2011}

Allgemein jedoch, k"onnen Unternehmen Soziale Medien, mittels Analysewerkzeugen, als Verhaltens- oder Kompetenzdatenbanken ihrer eigenen Mitarbeiter verwenden. Durch die Analyse von Meinungen und Verhaltensmustern von Mitarbeitern k"onnen versteckte arbeitstechnische Aspekte und informelle Abl"aufe innerhalb des Unternehmens beleuchtet werden. Solche Informationen k"onnen dann in weitere Folge im internen Human-Ressource (HR) Management oder bei der Definition von Business-Prozessen verwendet werden. \cite{Sinha2012}

Das voranschreiten von Cloud-Technologien erm"oglicht die Entwicklung von internen Abl"aufen in eine andere Richtung. Die Cloud erm"oglicht das Auslagern von It-Infrastruktur und anderen Aktivit"aten um diese nur noch als ''Service'' in das Unternehmen einzubinden. Dynamische Skalierung um Spitzenlasten abzufangen ist ein weiterer Vorteil dazu. Diese Tendenz zum Auslagern kann auch auf ganze Business-Prozesse oder das Business-Prozess-Management angewendet werden. Mittels ''Business-Process- as - a - Service'' (BPaaS) k"onnen interne Prozesse teilweise oder sogar komplett ausgelagert werden. \cite{Stoitsev2012}

Bei einem hohen Grad an Automatisierung k"onnen Business-Prozesse teilweise komplett in die Cloud ausgelagert werden. Durch dynamisches Skalieren und Business-Prozess "as-a-service" Clouds k"onnen Kunden eigens zugeschnittene Prozesse angeboten werden, welche dann in der Cloud durchgef"uhrt werden. Dieser Ansatz muss jedoch mit klaren Kontrollmechanismen und Audits kombiniert werden um die Akzeptanz im Unternehmen zu gew"ahrleisten. Die exakte Kontrollierbarkeit der Prozesse zu gew"ahrleisten ist in diesem Fall die gr"o"ste Herausforderung. \cite{Accorsi2011}