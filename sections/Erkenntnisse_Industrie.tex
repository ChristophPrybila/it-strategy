Im heutigen digitalen Zeitalter ist es auch zunehmend wichtiger geworden die Erkenntnise von gro"sen Industrieanalysten f"ur die Strategie im Bereich der Informationstechnologien eines modernen Unternehmens in Betracht zu ziehen. Zu den heutigen gro"sen Marktf"uhren dieser Branche z"ahlen Firmen wie Forrester Research Inc., Gartner Inc., IDC Corporate oder auch Frost \& Sullivan. Jeder der genannten Firmen besch"aftigt mehr als 1000 Analysten, die weltweit f"ur ihre Kunden Trends erforschen, den globalen Markt analysieren sowie Prognosen f"ur die Zukunft erstellen. Deren Forschungen und Analysen sollen dabei in Hinblick auf die IT Strategie eines modernen Unternehmen als Entscheidungsunterst"utzung dienen um in der rapide wachsenden und kontinuerlich wandelnden IT Branche wettbewerbsf"ahig zu bleiben sowie f"ur die zuk"unftigen Herausforderungen optimal ger"ustet zu sein. 

Bevor begonnen wird ein Konzept f"ur eine IT Strategie zu erstellen bzw. sp"ater Schritt f"ur Schritt umzusetzen, m"ussen zuerst die Werte, Visionen und Ziele eines Unternehmens auskommuniziert und verinnerlicht werden. Jede technologische Strategie beruht auf der Unternehmensstragie, daher ist es in dieser Hinsicht besonders von Vorteil, diese beiden Vorgehen auf einander abzustimmen um Komplikationen, die insbesondere das Erreichen der Ziele erschweren, zu vermeiden.
 
Forrester Research Inc. bezeichnet die Wechselwirkung von Technologie und Gesch"aftsstrategie als Business Technology Strategy. Im heutigen digitalen Zeitalter ist die Effizienz einer Unternehmensstrategie stark abh"angig von der technologischen Komponente. Firmen k"onnen nur l"angerfristig erfolgreich sein und ihr volles Potenzial aussch"opfen, wenn sie die n"otigen Kompetenzen besitzen, technologische Ressourcen und Innovationen komplett in ihre Gesch"aftsstrategie zu integrieren. \glqq Business Technology Strategy must start with business and end with business. \grqq  

Der traditionelle Ansatz des Wasserfallmodells zur Entwicklung einer Strategie eines Unternehmen gilt schon lange als general"uberholt und w"urde den Herausforderungen und Anforderungen der heutigen Zeit nicht gewachsen sein. Stattdessen sollen die Entscheidungstr"ager eines Unternehmens einen agilen und iterativen Ansatz verfolgen. Die Marktverh"altnisse "andern sich "au"serst h"aufig und sehr rasant, umso wichtiger ist es hier schnell auf die Ver"anderungen zu reagieren und die Strategie entsprechend der neuen Anforderungen zu adaptieren\cite{ForresterResearch}. 