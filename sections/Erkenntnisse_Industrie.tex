Im heutigen digitalen Zeitalter ist es auch zunehmend wichtiger geworden die Erkenntnise von großen Industrieanalysten für die Strategie im Bereich der Informationstechnologien eines modernen Unternehmens in Betracht zu ziehen. Zu den heutigen großen Marktführen dieser Branche zählen Firmen wie Forrester Research Inc., Gartner Inc., IDC Corporate oder auch Frost \& Sullivan. Jeder der genannten Firmen beschäftigt mehr als 1000 Analysten, die weltweit für ihre Kunden Trends erforschen, den globalen Markt analysieren sowie Prognosen für die Zukunft erstellen. Deren Forschungen und Analysen sollen dabei in Hinblick auf die IT Strategie eines modernen Unternehmen als Entscheidungsunterstützung dienen um in der rapide wachsenden und kontinuerlich wandelnden IT Branche wettbewerbsfähig zu bleiben sowie für die zukünftigen Herausforderungen optimal gerüstet zu sein. 

Bevor begonnen wird ein Konzept für eine IT Strategie zu erstellen bzw. später Schritt für Schritt umzusetzen, müssen zuerst die Werte, Visionen und Ziele eines Unternehmens auskommuniziert und verinnerlicht werden. Jede technologische Strategie beruht auf der Unternehmensstragie, daher ist es in dieser Hinsicht besonders von Vorteil, diese beiden Vorgehen auf einander abzustimmen um Komplikationen, die insbesondere das Erreichen der Ziele erschweren, zu vermeiden.
 
Forrester Research Inc. bezeichnet die Wechselwirkung von Technologie und Geschäftsstrategie als Business Technology Strategy. Im heutigen digitalen Zeitalter ist die Effizienz einer Unternehmensstrategie stark abhängig von der technologischen Komponente. Firmen können nur längerfristig erfolgreich sein und ihr volles Potenzial ausschöpfen, wenn sie die nötigen Kompetenzen besitzen, technologische Ressourcen und Innovationen komplett in ihre Geschäftsstrategie zu integrieren. Eine Business Technology Strategy fängt mit Business an und endet auch mit Business. 

Der traditionelle Ansatz des Wasserfallmodells zur Entwicklung einer Strategie eines Unternehmen gilt schon lange als generalüberholt und würde den Herausforderungen und Anforderungen der heutigen Zeit nicht gewachsen sein. Stattdessen sollen die Entscheidungsträger eines Unternehmens einen agilen und iterativen Ansatz verfolgen. Die Marktverhältnisse ändern sich äußerst häufig und sehr rasant, umso wichtiger ist es hier schnell auf die Veränderungen zu reagieren und die Strategie entsprechend der neuen Anforderungen zu adaptieren\cite{ForresterResearch}. 