
\glqq An effective IT strategy that forms an integral part of the
business strategy is imperative ...\grqq

In einer durchgef�hrten Studie wurde gezeigt, dass es zwar den Versuch gibt eine IT-Strategie zu formulieren, jedoch stellt sich in der Praxis meistens heraus, dass diese nicht sehr effektiv ist. Mehrere Gr�nde werden daf�r sind: keine Koppelung mit business processes, keine Koppelung mit der Unternehmensstrategie, die organisatorischen Bereiche des Unternehmens wurden nicht richtig eingebunden und es gab eine ineffektive Koordinierungsstelle. Des Weiteren wurde gezeigt, dass sich die IT-Strategie zwar mit den technischen Aspekten befasst, jedoch zu wenig mit den organisatorischen Aspekten.
\cite{808363}
\\

Die zwei gr��ten H�rden in Unternehmen heutzutage sind, dass Verstehen des Wertes der IT und die richtige Ausrichtung der IT im Unternehmen. Oft kommt es auch vor, dass die IT-Strategie sehr Technik lastig ist, und es f�r Gesch�ftspartner schwer ist diese zu verstehen.
\cite{4339647}
\\

Die Umwelt von Unternehmen ver�ndert sich stetig mit der Weiterentwicklung bestehender und neuer Technologien. Hierf�r ist es wichtig, dass Unternehmen ihre Gesch�ftsprozesse anpassen und entsprechend ver�ndern. Demnach muss man darauf achten, dass Unternehmen m�glichst flexible Management-Informationssysteme besitzen, um immer die aktuellen Modelle des Managements abbilden zu k�nnen. Eine IT-Strategie hat den Zweck Informationssystem zu verbessern und bessere Dienste f�r die Management-Strategie zu leisten.
\cite{5304653}
\\

In den letzten Jahren ist eine interessante Transformation bei IT-Systemen zu beobachten. Es werden von vielen gro�en und Namhaften IT-Unternehmen Dienste �ber das Internet(eine sogenannter Web-Service) angeboten, die von Unternehmen gekauft werden k�nnen. Dies f�hrt die Unternehmen weg davon all ihre Hardware und Software selbst zu verwalten. Des Weiteren erf�hrt das Unternehmen durch diese neue Strategie einen immensen Kostenvorteil.
\cite{Strategy_Harvard}
\\

Unternehmen in allen Sektoren werden immer abh�ngiger von einer Informations- und Kommunikationstechnolgie, aufgrund einer immer und stetig weiterentwickelnden IKT-Technologie. Dennoch werden die Gesch�ftserwartungen der IT oft nicht eingehalten, da es Probleme bei der Durchf�hrung der IT-Strategie und schlechte Kontrollmechanismen gibt.
\cite{Velitchkov:2008:ISE:1500879.1500955}