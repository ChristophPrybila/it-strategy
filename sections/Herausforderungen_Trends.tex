\glqq The financial services industry has been transformed substantially over the last decade.\grqq
\cite{1508397} 
\\

Die immer gr��er werdende Vorherrschaft von Smartphones und dem Internet, ist eine gute Grundlage f�r das Wachstum von mobile banking. Normalerweise verwendet man mobile banking f�r das Anzeigen des aktuellen Saldos des Kontos oder f�r �berweisungen, jedoch kann man kein Bargeld beheben. Der Vorteil von mobile banking ist die Kostenersparnis f�r die Banken, da der Kunde ohne zu tun von Bankangestellten oder des Besuchs einer Bankfiliale viele Bankgesch�fte erledigen kann. Au�erdem erh�ht es den Umsatz von Anbietern mobiler Dienste, da man f�r die Nutzung mobiler Dienste Geld verlangen kann. Ein weiterer Vorteil von mobile banking ist die M�glichkeit der Erschlie�ung neuer M�rkte, bei der �rmeren Bev�lkerung.
\cite{5723885}
\\

Die immer weiter wachsende Bev�lkerung von  China bietet gro�e Herausforderungen und Chancen f�r den Bankensektor. Insgesamt gab es 2007 210 Millionen Internetuser, davon nutzten 19,2\% internet banking. Wichtig hierbei ist, dass man dem Kunden vermittelt, wie einfach internet banking zu benutzen ist.
\cite{4680308}
\\

Die rasche Entwicklung von mobile-commerce Applikationen, wie mobile banking revolutionieren den Bankensektor. Mobile banking erlaubt den Zugriff auf die Bankdienste zu jeder Zeit und an jedem Ort. Jedoch sind die mobilen Endger�te anf�llig f�r Angriffe durch Dritte, deshalb ist die Sicherheit ein wesentliches Thema in diesem Bereich. Es ist wichtig geeignete kryptografische Verfahren beim Transfer der Daten einzusetzen, um die Daten der Kunden zu sch�tzen.
\cite{4812913}
\\

Besonders in Schwellenl�ndern k�nnte mobile banking zum prim�ren Kanal zwischen Bank und Kunde werden. Es wird auch erwartet, dass m-banking das Kreditkartensystem und das klassische netbanking ersetzen wird. Auch der Druck auf der Kundenseite wird in den n�chsten Jahren zunehmen, da es immer billigere Endger�te am Markt gibt, die f�hig sind mobile-banking zu unterst�tzen.
\cite{5169360}
\\

Eine gro�e Herausforderung in Schwellenl�ndern ist jedoch beim mobile banking die gr��ere Verbreitung, da es unterentwickelte Telekommunikationseinrichtungen und eine geringe Bekanntheit von m-banking gibt. 
\cite{5286659}

