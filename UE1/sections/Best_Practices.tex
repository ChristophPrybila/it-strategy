Lange Zeit galt das Gesch"aftsmodell der Universalbank als vorbildlich. Jedoch zeigten sich durch die Entwicklung der Globalisierung und der letzten Finanzkrise einige Schw"achen hinsichtlich von Flexibilit"at, Transparenz der Wertsch"opfung und Kundenkommunikation. Dadurch r"ucken schlankere Gesch"aftsmodelle wieder verst�rkt in den Fokus. \cite{Bieger2011} 

Grundlegende Reorganisationen begannen mit ''Lean-Management'', mussten jedoch von weiteren IT Ma"snahmen unterst"utzt werden. \cite{Fischer2013} Der IT kommt in der globalen Vernetzung der Banken eine Schl"usselrolle zu. Es stellen sich eine Reihe von Herausforderungen um eine flexible IT-Architektur zu schaffen. Dies beginnt mit der Erstellung einer gemeinsamen Datenbasis bis zum Umsetzung eines effektiven Schnittstellenmanagements. \cite{Stahl2003}

Urspr"unglich eigens entwickelte Infrastruktur muss durch anwendungsneutralen Systeme ersetzt werden. Dadurch k"onnen Effizienzsteigerungen erreicht werden, wenn die Anzahl der IT-Schnittstellen zu den eigenen Mitarbeitern reduziert wird. Gleichzeitig k"onnen einheitliche und modulare Systeme die Kundenorientierung von Prozessen erh"ohen.\cite{Moebus2000}

Die Banken"ubergreifende Kommunikation stellt Aufgrund der vielen Legacy Systeme welche innerhalb der gro"sen Unternehmen zum Einsatz kommen weiterhin eine gro"se Herausforderung dar. Schindler (2008) versucht diese Problematik mittels dem INPAR Systems zu veranschaulichen und zu untersuchen. Er kommt zum Schluss, dass die Anwendung von aktuelleren Technologien f"ur die Kommunikation und Wartbarkeit wesentliche Verbesserungen bringen w"urde. Er schl"agt einen Kombinierten Einsatz von XML als Datenformat als auch von Web Services als Kommunikationstechnologie aufgrund ihrer weiten Verbreitung und Akzeptanz vor. Dies sollte Banken zu einigen Vorteilen im Bezug auf Flexibilit"at und Erweiterbarkeit verhelfen.\cite{Schindler2008}

Eine M"oglichkeit zur Realisierung und Unterst"utzung einer solchen Strategie zeigte sich im konservativen Einsatz von SOA zur agilen Gestaltung von Gesch"aftsprozessen. Baskerville et. al. untersucht dies an skandinavischen und schweizer Banken.\cite{Baskerville2010} 


Allgemein jedoch, sollte die Einf"uhrung von agiler Business-Prozess Verwaltung zu einer Reduktion und nicht zu einer Erh"ohung der IT-Komplexit"at f"uhren. Dies wurde am Beispiel der Credit Suise demonstriert. \cite{Kurmann2007}

In vielen neuen Ma"rkten ist Internet-Banking jedoch noch nicht so stark verbreitet wie etwa in Westeuropa. Wirtschaftliche aber vor allem auch kulturelle Faktoren bestimmen die Akzeptanz von eingef"uhrten Internet-Banking Produkten. Dies wird an den Beispielen von Nigeria\cite{Sarlak2010} und Mauritius\cite{Juwaheer2012} beschrieben.

