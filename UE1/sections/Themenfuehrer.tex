
Eines der wichtigsten Themen im Bankensektor ist die Sicherheit der IT-Infrastruktur. Aus diesem Grund wurden in einigen Industrienationen eigene Abteilungen eingerichtet, welche die Risiken von IT-Systemen �berwachen. Wie zum Beispiel die American Monetary Administration oder die Hong Kong Monetary Authority. Amerika spielt in diesem Bereich eine besondere Vorreiterrolle, waren sie doch das erste Land welches ein IT-Risikomanagement eingerichtet haben.
\cite{5592630}
\\

Ein Problem bei IT basierten Bank-Applikationen im allgemeinen ist die Authentifizierung des Kunden. Da es durch Hackerangriffe passieren kann, dass Passw�rter gestohlen werden, wird von einer Expertengruppe vorgeschlagen das Passwort des Kunden mit einem verbesserten Steganographie-Verfahren zu verstecken. Dieser verbesserte Algorithmus verwendet mehr Zielpixel, sodass die Unwahrnehmbarkeit steigt. Darauf folgend werden die Zielpixel aufgeteilt zwischen der Bank und dem Kunden, um das Passwort nun rekonstruieren zu k�nnen ben�tigt man beide Teile. Zudem hat dieses Verfahren den Vorteil, dass man mit einer Kompromittierung der Bank oder des Kunden nicht das Passwort rekonstruieren kann.
\cite{6203923}
\\

Besonders die Entwicklung von Software die im Finanz- und Bankenbereich eingesetzt wird muss speziellen Qualit�tsanforderungen gen�gen, da diese sicherheitskritische Software in einem Fehlerfall zu einem signifikanten Verm�gensverlust f�hren kann.
\cite{6407380}
\\

Eine Vorreiterrolle in Sachen mobile-payment sind Schwellenl�nder. Tom Standage, hat in einem Artikel des Economist gesagt, dass es einfacher ist mit einem Mobiltelefon in Nairobi zu bezahlen, als in New York. Kenia hat mit Stand Februar 2012 18 Millionen mobile-payment Kunden. Generell wird f�r den Asien-Pazifik Raum ein gro"ses Wachstum vorhergesagt.
\cite{6248655}
\\

In China ist der Bankensektor gerade im Wandel, die gr�"sten Banken Chinas bewegen sich weg von den weltweit gr�"sten Privatkundenbanken hin zu gut integrierten Finanzdienstleistern mit hochverf�gbaren Bank-Applikationen und big data business. Jedoch sind die momentan genutzten Rechenzentren der Banken nicht daf�r ger�stet. Eines der schwierigsten Ziele ist das Ausrollen des Next Generation Banking Systems. Dies ist die gr�"ste Umstellung f�r Banken mit �ber 5000 physikalischen Servern und noch mehr virtualisierte Ressourcen in einer privaten Cloud. Eine der gr�"sten Banken, die an dieser Umstellung beteiligt sind ist, die China Industrial and Commercial Bank of China Limited(ICBC) besitzt insgesamt 282 Millionen Kunden.
\cite{6649734}

