Lange Zeit war es gang und g�be im Bankensektor auf aufwendige Eigenl�sungen zur�ckzugreifen. Mit den immer steigenden Anforderungen an Banken im europ�ischen Raum, zusammen mit immer strikter werdenden gesetzlichen Regelungen, ist es f�r Banken zunehmend schwieriger auf all diese �nderungen in angemessener Zeit und mithilfe der gegebenen Ressourcen zu reagieren. F�r die meisten Banken ist die Verwendung und st�ndige Anpassung einer (teils) komplett veralteten Eigenl�sung nicht mehr leistbar, auch weil das ben�tigte Know-How in Form von Mitarbeitern zunehmend am Ende der Lebensarbeitszeit ist. Dadurch werden moderne Standardsoftware und Outsourcingstrategien immer interessanter.~\cite{Berger2012}\\
Auch bei den Sourcingstrategien vollzieht sich seit den letzten Jahren ein Wandel in Richtung \textit{Multiple-Sourcing}. Dies war im Finanzdienstleistungssektor lange ein Tabuthema, da die Integrationskosten f�r zu hoch erachtet wurden und den entstehenden Effekt der erh�hten Ausfallsicherheit zu rechtfertigen. Doch mit dem derzeitigen Stand der IT-Technik und der zunehmenden Verbreitung von Standards durch den Einsatz von Standardsoftware ist es laut Buhl ~\cite{Buhl2011} auch in der Finanzdienstleisterbranche denkbar und von Vorteil eine \textit{Multiple-Sourcing} Strategie einzusetzen.