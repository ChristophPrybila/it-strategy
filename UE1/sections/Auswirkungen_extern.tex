Der Mittelpunkt von Smarten Technologien und Sozialen Medien ist die Interaktion von deren Nutzern. Diese Technologien und Medien erleben seit Jahren ein rasantes Wachstum. In seinen Anf"angen brauchte das Internet nur vier Jahre um 50 Millionen Nutzer zu erreichen. Facebook wuchs um 50 Millionen Nutzer in nur einem halben Jahr. \cite{Nair2011} Nutzerinteraktion bezieht sich dabei einerseits auf den Konsum von Inhalten, aber auch das Erstellen von Inhalten selbst. \cite{Alt2012}

Durch die Abwanderung zu neuen digitalen Medien verlieren die urspr"unglichen Printmedien, Radio oder Fernsehen zunehmend an Bedeutung. Echtzeit-Information wird ein integrales Element f"ur Kundenverhalten. \cite{Hennig2010}

Durch diese fortschreitende Verbreitung von mobilen Technologien, wird Kommunikation auch zusehends Orts- und Situationsabh"angig. \cite{Alt2012} Diese neuen M"oglichkeiten der Interaktion ver"andert auch das verhalten von Kunden. Dies gilt vor allem f"ur die Art und Weise wie Information "uber Unternehmen und Produkte gesammelt und ausgetauscht werden. \cite{Hennig2010} Smarte Technologien und Soziale Applikationen geben Konsumenten die verschiedensten M"oglichkeiten aktiv Informationen "uber Produkte und Dienste anzubieten. \cite{Foster2010} Durch Plattformen wie Ebay oder Amazon werden Konsumenten auch zu Verk"aufern ihrer eigenen Produkte.

Unternehmen m"ussen sich durch dieses ver"anderte Kommunikationsverhalten auf neue  Herausforderung im Bereich des Customer-Relationship-Managements (CRM) einstellen. Der mittelbare Kundenkontakt ver"andert sich. Au"s�endienstmitarbeiter, Kundenberater oder Call-Center-Agenten verlieren an Bedeutung. Kunden erwarten einen unmittelbaren mit den Unternehmen. \cite{Alt2012} Sie werden zu aktiven und stark vernetzten Partnern, welche auch die Rolle von Anbietern und Produzenten einnehmen k"onnen. Dies macht es f"ur Unternehmen schwieriger das eigene Marken-Image und das Klima von Kundenbeziehungen zu kontrollieren \cite{Hennig2010}

Es ergeben sich viele neue Risiken f"ur Unternehmen im Umgang mit sozialen Medien. Kunden geben den Erfahrungen und Meinungen anderer Konsumente h"oheren Stellenwert und Glaubw"urdigkeit gegen"uber der Unternehmenskommunikation. Eskalierte Diskussionen in sozialen Medien k"onnen starke negativen Auswirkungen auf Unternehmen haben diese sp"at oder gar nicht daran teilnehmen. \cite{Alt2012}

Gleichzeitig gibt es auch viel Potential f"ur neue Kommunikationsformen zwischen Kunden und Unternehmen. Unternehmen bekommen die M"oglichkeit direkt mit ihren Kunden zu interagieren um sich "uber Kampagnen oder Probleme auszutauschen. Oftmals wird eine solche direkte Interaktion von den Kunden auch erwartet. \cite{Alt2012}

Besonders f"ur Unternehmen mit Endkundenkontakt ist die Nutzung des SocialWeb eine wettbewerblichen Notwendigkeit. Gleichzeitig darf der Einsatz von neuen Medien, "altere Konsumenten nicht ausschlie"sen. Der Kontakt zu Konsumenten aus "alteren Bev"olkerungsschichten darf nicht vernachl"assigt werden. \cite{Coughlin2007}