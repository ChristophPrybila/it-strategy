
\textbf{Sourcingstrategie}\\
Die Workstations und Betriebssysteme sollen aufgrund des schnellen Wandels, Auslagerung der Wartung und steuerlichen Vorteilen von einem Anbieter gemietet werden. Dabei ist darauf zu achten, dass aufgrund der Gew�hlten Strategie nur Thin-Clients ben�tigt werden und dabei auf eine m�glichst billige (open-source) Betriebssysteml�sung zur�ckgegriffen werden soll.\\
Da f�r Kernprozesse eigene Server betrieben werden m�ssen, ist die daf�r ben�tigte Hardware (Server, Router, Switches, usw.) ebenfalls anzumieten. Dabei ist darauf zu achten entsprechende Service Level Agreements abzuschlie"sen, die die ben�tigte Verf�gbarkeit garantieren.\\
Auf diesen unternehmensinternen Servern soll auch die \textit{Private Cloud} aufgespielt werden. Diese \textit{Private Cloud} muss von einem externen Dienstleister implementiert und gewartet werden. Wenn die M�glichkeit besteht, Hardware und Software von einem Unternehmen zu beziehen, sollte diese M�glichkeit wahrgenommen werden um ein problemloses Zusammenspiel zu garantieren.\\
Ziel dieser \textit{Private Cloud} ist es auch die einzelnen Standorte technologisch zusammen zu f�hren. Deswegen soll auch der Internetanschluss f�r alle Standorte von einem ISP verwaltet werden.\\
Sofern m�glich, wird jegliche Software die f�r Supportprozesse ben�tigt wird von einer externen Firma implementiert und gewartet. Nichts desto trotz muss f�r die Kernkomponenten der ben�tigten Software ein unternehmensinternes Entwicklerteam eingerichtet werden, welches f�r die Erstellung und Wartung ebendieser zust�ndig ist.\\
Bei allen Applikationen und Dienstleistungen die von Dritten zugekauft werden, muss besonderes Augenmerk auf die Sicherheit und den Datenschutz gelegt werden.\\
Anwendungen die unternehmensweit auf jeder Workstation eingesetzt werden (zum Beispiel Office- und Email-Applikationen) sollen standortunabh�ngig von der Zentrale gekauft werden und alle Lizenzen und Updates auch von ebendieser verwaltet und automatisch eingespielt werden.\\
F�r Mobile Banking, SmartCard Banking und Embedded Banking Applikationen ist (jeweils) ein Externer Entwickler zu w�hlen, der hohe Sicherheitsstandards erf�llt und Erfahrung mit den ben�tigten Lizenzen und Vertr�gen f�r den Europ"aischen Zahlungsraum hat.\\

\textbf{Investmentstrategie}\\
Hier wird der Weg der geringsten Kapitalbindung gegangen. Dadurch ist auch die Anfangsinvestition relativ gering und die Kosten sind auf die Laufzeit der gemieteten Hard- und Software aufgeteilt. Dadurch ergibt sich der Vorteil, dass m�glicherweise abschreckende, initiale Kosten einer neuen IT-Strategie ausbleiben.\\
