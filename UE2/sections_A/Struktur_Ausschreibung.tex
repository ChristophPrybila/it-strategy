Folgende Form sollte eine Ausschreibung vorweisen: \cite{Works_Jan_2011_Final.pdf}

Teil 1 - Ausschreibungsprozedere
\begin{itemize}
	\item Anleitung f�r Bieter: Dieser Teil soll den Bietern helfen ein Angebot zu erstellen. Es werden au"serdem Informationen �ber die Einreichung und Evaluierung geboten.
	\item Evaluierungs- und Qualifikationskriterien: Hier werden die Kriterien spezifiziert, die benutzt werden um die Angebote der Bieter zu bewerten.
	\item Formulare: In diesem Bereich werden die Forumlare aufgelistet, welche ausgef�llt werden m�ssen um am Verfahren teilzunehmen.
	\item Berechtigte L�nder: Die L�nder die berechtigt sind an der Ausschreibung teilzunehmen.
\end{itemize}

Teil 2 - Anforderungen

\begin{itemize}

	\item Hier wird detailliert beschrieben was von dem Bieter entwickelt werden soll. Dieser Abschnitt enth�lt Spezifikationen, Abbildungen, etc. um eine m�glichst genaue Beschreibung zu geben.

\end{itemize}

Teil 3 - Konditionen

\begin{itemize}

\item Generelle Konditionen

\item Spezielle Bestimmungen des Vertrags

\end{itemize}

Des Weiteren ist es sinnvoll die Sprache der Angebote auf die Ausschreibung festzulegen zum Beispiel Deutsch und/oder Englisch. Eine Begrenzung der Seitenzahl der Angebote ist nicht notwendig, jedoch ein Hinweis pr�zise und knappe Formulierungen zu verwenden. Das Angebot des Bieters sollte au"serdem einen Work Breakdown beinhalten. Ziel hierbei ist es eine systematische Beschreibung der Vorgangsweise bei der Entwicklung zu geben, welche durchgef�hrt wird um den Anforderungen der Ausschreibung zu entsprechen. Dies erm�glicht eine angemessene �berpr�fung des Angebots, der Kosten und der vorgeschlagenen Planung.
\cite{gce.html}

Es ist empfehlenswert die Identit�t des Anbieters zu �berpr�fen und zu diesem Zweck einen Gewerberegisterauszug zu verlangen und ein Leumundszeugnis der oder des Gesch�ftsf�hrers. Falls der Anbieter eine nat�rliche Person ist muss eine Ausweiskopie beigelegt werden.
\cite{ewsi_terms_of_reference.pdf}