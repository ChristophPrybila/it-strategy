Cloud computing ist der neueste Trend in IT-Outsourcing. Viele existierende Technologien und Features wurden umbenannt um als Teil dieses Trends zu erscheinen. Viele Webservices und Online Storages verkaufen sich auf diese Weise besser. Es ist immer schwieriger geworden eine konkrete Definition der ''neuen'' cloud computing Technologien zu geben. Die eigentlichen Neuerungen und Intentionen liegen im dynamischen Skalieren der genutzten Services und Features. Gleichzeitig wird als Bezahlschema ''pay-per-use'' angeboten. \cite{Boehm2009} 

Den Vorteilen einer solchen Architektur stehen die folgenden Probleme und Herausforderungen gegen"uber. \cite{Motahari2009}

\begin{itemize}

	\item Ein Kunde verliert die direkte Kontrolle "uber die eingesetzte Hardware und Software.
	
	\item Es besteht eine erh"ohte Abh"angigkeit zum Cloud-Anbieter. F"ur den Kunden entsteht durch diese enge Kopplung ein Risiko. Die Probleme des Cloud-Anbieters werden die Probleme des Kunden.
	
	\item Die Kompatibilit"at zu Software au�erhalb der Cloud wird verringernd und/oder die Auswahl von Software Technologien in der Cloud ist eingeschr"ankt.

	\item Mangels fehlender Transparenz bei dem Einsatz der Sicherheitsma�nahmen und bei der Behandlung von Sicherheitsvorf�llen besteht ein erh"ohtes Sicherheitsrisiko. Die gemeinsame Verwaltung mehrerer Kunden im selben Gesamtsystem stellt ein weiteres Risiko f"ur den einzelnen Kunden dar. \cite{Ardelt2011}
	
	\item Eine Reihe an ungekl"arten Rechtsfragen bez"uglich der dynamischen Lokalit"at von Services und Daten ist eine Herausforderung f"ur die Festlegung der einzusetzenden Rechtsbasis und der damit einhergehenden Bestimmung f"ur Lizenzen und Urheberrecht.\cite{Picot2001}

\end{itemize}

Weiteres stellt sich f"ur neue Anwender, welche zu Cloud Technologien migrieren wollen, die folgenden Fragen. \cite{Motahari2009}

\begin{itemize}

	\item Welche Funktionen sowie Business-Prozesse sollen in welcher Reihenfolge in die Cloud ausgelagert werden?
	
	\item Wie kann ein solcher "Ubergang, vor allem in Hinsicht auf Legacy Systeme und deren Anforderungen, flie�end erfolgen?
	
	\item Welche Anbieter bietet die geeigneten Services welche alle notwendigen Anforderungen erf"ullen?	

\end{itemize}