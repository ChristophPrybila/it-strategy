TBW Kriterien Outsourcing\\
Beim Thema Outsourcing bzw. Managed Services gibt es mehrere Faktoren, die im Entscheidungsprozess ber"ucksichtigt werden sollten. Abgesehen von Preis und Qualit"at des ausgelagerten Services bzw. Produktes, welche letztendlich immer eine entscheidende Rolle spielen, sollen Outsourcing Strategien einen bestm"oglichen Vorteil f"ur ein Unternehmen bringen und auch die Produktivit"at innerhalb des Konzerns steigern. Nach dem Paper von Yang und Huang \cite{Yang2000225} sollen bei einer gewissenhaften Outsourcing Entscheidung 5 Faktoren bzw. Dimensionen verwendet werden: Management, Strategie, Technologie, "Okonomie und Qualit"at. Hinzu kommen unterschiedliche Attribute, die je nach Entscheidungsgrundlage eines Unternehmens bei den erw"ahnten Faktoren ber"ucksichtigt werden.

\begin{itemize}
	\item Management
	
	Hier kommt der Gesch"afts"f"uhrung eine wichtige Rolle um m"ogliche Kommunikationsbarrieren und Performanceprobleme in IT-Abteilungen zu erkennen und zu beseitigen. Das Management hat die M"oglichkeit gewisse IT-Funktionen bzw. die gesamte IT-Abteilung auszulagern um so die Performance nachtr"aglich zu steigern.
	\item Strategie
	
	Hier sollte sich die Firma ganz auf ihre Kernkompotenzen fokussieren und Aktivit"aten, die nicht dazu z"ahlen, auslagern. Zudem besteht die M"oglichkeit mit Anbietern eine strategische Zusammenarbeit einzugehen um neue Produkte und Technologien schneller zu entwickeln und auf dem Markt zu bringen(Product Time-to-Market). Au"serdem wird auch das strategische Risiko f"ur das Unternehmen minimiert, da es mit dem Outsourcing Partner geteilt wird.
	
	\item Technologie
	
	Der schnellste und effizienteste Weg einen Zugang zu den neuesten IT-Technologien zu erhalten, ist Outsourcing. F"ur die Mitarbeiter des Unternehmens entstehen M"oglichkeiten mit neuen Technologien in Kontakt zu kommen und diese kennen zu lernen.
	
	\item "Okonomie
	
	Das Hauptaugenmerk in der "Okonomie liegt darin, die Implementierungs- und Wartungskosten von IT-unterst"utzten Anwendungen zu senken. Zudem obliegt auch dem Outsourcing Partner die Verantwortung bei Beschaffung von Hardware, Software und Personal. Dadurch werden auch die fixen Kosten zu variablen Kosten, da die Kompotenzen dem Anbieter "ubertragen werden und das Ergebnis ist eine h"ohere Flexibilit"at in finanzieller Hinsicht. 
	
	\item Qualit"at
	
	F"ur den Qualit"atsaspekt sollte es einen signifikanten Unterschied in der Servicequalit"at geben, welches vom Outsourcing Partner angeboten wird, im Vergleich zu der Insourcing-Strategie. Um hohe Zuverl"assigkeit und gute Service Qualit"at zu gew"ahrleisten, sollte die Firma Performance Ziele und Service Level in Vertr"agen miteinbinden.
\end{itemize}
	
