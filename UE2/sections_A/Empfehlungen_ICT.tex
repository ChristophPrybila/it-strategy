Der Gesamteffekt der cloud-basierten Dienste kann zwar noch nicht voll abgesch"atzt werden, jedoch sollte man laut Deloitte folgende Entwicklungen beobachten. In Zukunft k"onnten sich cloud-basierte infrastructure as a service(IaaS) viel billiger erweisen als traditionelle IT-Umgebungen. Damit IT-Outsourcing Anbieter diesem Trend standhalten k"onnen m"ussen sie sich anpassen um cloud-"ahnliche Performance zu einem cloud-"ahnlichen Preis anbieten zu k"onnen.
Des Weiteren ist ein starker Anstieg im Application development zu erkennen. Aufgrund der nun mehr verf"ugbaren cloud-basierten Entwicklunsplattformen ist es sehr einfach m"oglich, ohne hohe Investitionen zu t"atigen, Applikationen zu entwickeln.
John Tweardy ein Direktor von Deloitte Consulting LLP. meint, dass Trends zu erkennen sind von alten Systemen hin zu neuen cloud und mobil basierten Systemen. Eines ist jedoch klar: Cloud-computing wird f"ur das Outsourcing eine wichtige Rolle spielen.
\cite{L339}

Laut Robert Hillard(Deloitte) wird IT-Outsourcing zur"uckgehen um die Kompetenzen wieder zur"uck ins Unternehmen zu bringen. Matthew Oostveen von IDC behauptet au"serdem, dass die Vertragsl"ange von Outsourcing-Vertr"agen von den momentan "ublichen 6-8 Jahren auf 3-5 Jahre zur"uckgehen wird.
Einer anderen Meinung ist Rolf Jester von Gartner, er meint IT-Outsourcing wird den Markt 2013 dominieren, jedoch wird cloud computing eine immer wichtigere Rolle spielen.
\cite{341}

Outsourcing wird 2014 um 4 Prozent wachsen laut Analysten von Ovum. Laut Thomas Reuner Chef-Analyst von Ovum gibt es zwar manche Personen, die das Ende von Outsourcing vorhergesagt haben bleibt es dennoch ein wichtiges strategisches Werkzeug.
\cite{OVUM342}

Speziell im US-Markt wird ein st"arkeres Wachstum von cloud-basierten Diensten erwartet als das herk"ommliche IT Outsourcing. Viele Firmen zeigen Interesse ihr CRM, HR und E-Mail Funktion zu SaaS-Providern out zu sourcen. Fr"uher war es jedoch "ublich für jeden Bereich einen eigenen Anbieter zu verwenden. Nunmehr kann man nur mit einem Anbieter alle Bereiche abdecken. Beispiele hierf"ur sind Salesforce, Workday oder Oracle Fusion.
Stanton Jones ein Analyst bei ISG erkennt einen Trend hin zu Infrastructure-as-a-Service(IaaS) in den IT-Outsourcing Vertr"agen. Man muss auch unterscheiden zwischen IT-Outsourcing(ITO) und Gesch"aftsprozess Outsourcing(BPO). Bei ITO geht es um Infrastructure-as-a-Service, wobei die Unternehmen Anbieter f"ur Rechenkapazit"at, Speicher oder Middleware suchen. Bei BPO sind Software-as-a-Service Angebote interessant, hier sind auch gr"o"sere "Anderungen sehr wahrscheinlich.
\cite{ISG17}
