Aufgrund der rasanten Verbreitung von Outsourcing Strategien werden Make or Buy Entscheidungen immer popul"arer und finden h"aufiger Anwendung auf Managementebene. Kein Unternehmen sollte eine L"osung mit internen Ressourcen entwickeln, wenn sie die M"oglichkeit hat, dieselbe L"osung kosteng"unstig von einem anderen Anbieter einzukaufen. Daher gilt es immer im Vorraus mit Entscheidungs- und Berechnungsmodellen zu arbeiten, welches als Unterst"utzungsgrundlage bei Make or Buy Fragen dienen. Es werden generell zwischen 3 Grundkategorien von methodischen Ans"atzen bei Make Or Buy Entscheidungen unterschieden: Monovariate, Bivariate oder Multivariate Modelle, abh"angig davon viele unterschiedliche Kriterien f"ur die Beurteilung ber"ucksichtigt werden.\cite{GeVo2009}

\begin{itemize}
	\item Monovariate Modelle
	
	Bei Monovariaten Modellen wird nur genau eine Variable beobachtet um einen m"oglichen Vorteil bei Make Or Buy Entscheidungen zu beurteilen. Speziell im Bereich Transaktionskostentheorie finden diese Art von Ans"atzen Verwendung. In der Praxis findet der Monovariate Ansatz f"ur die endg"ultige Entscheidung aber wenig Verwendung, da meist mehrere Dimensionen ber"ucksichtigt werden.
	
	\item Bivariate Modelle
	
	Bei Bivariaten Modellen werden zwei Variablen f"ur eine Make Or Buy Entscheidung ber"ucksichtigt. In der Praxis wird oft ein Matrix bzw. Portfolio Design verwendet um die Entscheidungsstruktur zu illustrieren. Anhand der Visualisierung bekommt die Managementebene eine gute "Ubersicht "uber die momentane Situations des Unternehmens. Da es in der Praxis meist mehr als zwei Variablen gibt und bei einem Bivariaten Ansatz die Komplexit"at des Entscheidungsfindungsprozess auf zwei Dimensionen reduziert wird, kann das Modell nur als Unterst"utzung f"ur das Management angesehen werden und nicht als klare Empfehlung f"ur eine richtige Auswahl.
	
	\item Multivariate Modelle
	
	Ans"atze bei denen mehr als zwei Variablen f"ur eine Make Or Buy Entschedung ber"ucksichtigt werden, nennt man Multivariate Ans"atze. Mehrere Ans"atze von unterschiedlichen Matrix Modellen werden mit den Transaktionskostenansatz kombiniert um eine neue Form der Bewertung zu erhalten. Anhand dieses Modells k"onnen auch die unterschiedlichen Kriterien nach Wichtigkeit geordnet werden. Die Multivariaten Ans"atze sind komplexer und anspruchsvoller als die Mono- und Bivariaten. Anhand dieser Modelle ist es auch m"oglich die Probleme von Mono- und Bivariaten Modelle zu l"osen aufgrund ihrer Flexibilit"at.
\end{itemize} 

	