Um aus der Menge von m"oglichen Anbieter-Kandidaten den besten Anbieter zu finden, sind die folgenden Schritte notwendig. \cite{Wan2013}

\begin{enumerate}

	\item \textbf{Identifikation der Key-Anforderungen}\\
Definition der eigenen Kriterien und Anforderungen. Abw"agung des Aufwandes zwischen der Wichtigkeit des Kaufgegenstandes und den Kosten und den M"oglichkeiten f"ur die m"oglichen Anbieter.
	
	\item \textbf{Definieren der Sourcing Strategie}\\
Eingrenzung durch Entscheidung zwischen den folgenden Strategien.

	\begin{itemize}
	
		\item single vs. multiple sourcing
		\item short-term vs. long-term contracts
		\item design support vs. operational support
		\item full-service vs. non-full-service suppliers
		\item domestic vs. foreign-based suppliers
		\item collaboration vs. arm's length relationship	
		
	\end{itemize}
	
	\item \textbf{Identifikation der potentiellen Anbieter}\\
Auf Basis der bisherigen Erfahrung und unter Verwendung von verf"ugbaren Informationsdatenbanken wie etwa Trade journals, Trade directories und Trade shows.
	
	\item \textbf{Eingrenzung dieser gefundenen Anbieterauswahl}\\
Entfernung von offensichtlich unpassenden Anbietern durch Risiko-Analysen, finanziellen Absch"atzungen und Evaluation von bisherigen Leistungen. 

	\item \textbf{Festlegung der Evaluationsmethode}\\
Evaluation auf Basis der Anbieterinformationen und externer Informationsquellen. Durchf"uhrung von Anbieterbesuchen und treffen. Auswahl einer branchen"ublichen oder wissenschaftlich akzeptierten Evaluationsmethode.
	
	\item \textbf{Durchf"uhrung der Festgelegten Evaluationsmethode}
	
	\item \textbf{Selektion des besten Anbieters und beginn der Vertragsverhandlungen}

\end{enumerate}