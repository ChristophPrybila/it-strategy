Mit jedem Tag gibt es mehr Software und Services am Markt. Um dieser Masse an Angeboten Herr zu werden ist es n\"otig, Anforderungen klar zu definieren und zu wissen welche Kriterien bei der Auswahl relevant sind. Laut Repschl\"ager~\cite{Repschl2013} sind die wichtigsten Zieldimensionen f\"ur Cloud Services:
\begin{itemize}
\item Flexibilit\"at
\item Leistungsumfang
\item Ausfallsicherheit \& Vertrauenswürdigkeit
\item Service \& Cloud Management
\item IT Sicherheit \& Compliance
\item Kosten
\end{itemize}
Aus diesen Zieldimensionen lassen sich im ersten Schritt die Anforderungen ableiten, welche durch bestimmten Bewertungskriterien (unterteilt in  Anbieter- und Dienstkriterien) zur endg\"ultigen Auswahl f\"uhren.\\

Bei der Auswahl von Applikationen verh\"alt es sich \"ahnlich, jedoch nicht ganz gleich. So gibt Kl\"upfel~\cite{Klupfel2007} folgende Kriterien an (zusammengefasst):\\
Schritt 1 - Auswahl nach:
\begin{itemize}
\item Technologie/Funktionale Abdeckung
\item Zuk\"unftige Bed\"urfnisse und Anbieterleistungen
\item Kostenrahmen
\end{itemize}
Schritt 2 - Auswahl nach:
\begin{itemize}
\item Exakte Kostenermittlung
\item L\"osungskompetenz
\item Pr\"asentation und Look \& Feel
\end{itemize}
Schritt 3 - Auswahl nach:
\begin{itemize}
\item Konditionen
\item Sicherheit
\item Verbindlichkeit
\end{itemize}
Nach diesen drei Auswahlschritten sollte das gew\"unschte Produkt gefunden sein und einem erfolgreichen Einsatz der ausgew\"ahlten Applikation nichts mehr im Wege stehen.