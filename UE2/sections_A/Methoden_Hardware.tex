In vielen F\"allen kommen qualitative Methoden zur Auswahl von Hardware zum Einsatz. Diese Methoden beruhen vor allem auf Expertenwissen und die richte Einsch\"atzung von Performancevorteilen in Relation zu den Mehrkosten.\\
Probleme kommen jedoch auf, wenn Hardwaresysteme zunehmend komplex werden, oder eine solide, faktische Argumentation f\"ur die getroffene Auswahl verlangt ist. In diesen F\"allen muss auf quantitative Methoden zur\"uck gegriffen werden wie zum Beispiel \textit{Logic Scoring of Preferences (LSP)} eine ist.\\
LSP wird von Dujmovic~\cite{Dujmovic1996} sehr gut vorgestellt und umfasst bei der Anwendung auf ein Hardwareproblem folgende Schritte:
\begin{itemize}
\item Machbarkeitsstudie
\item Spezifikation von Leistungsvariablen
\item Definition von elementaren Kriterien
\item Spezifikation der Pr\"aferenzenaggregationsstruktur
\item Anfordern von Angeboten
\item Erstellung von Angeboten 
\item Systemevaluierung und Selektion unter Ber\"ucksichtigung der Kostenpr\"aferenzenanalyse
\item Vertragsabschluss, Installation und Akzeptanztest
\end{itemize}
Im Zuge dieser Schritte kommt man auf ein mathematisches Entscheideproblem. Dieses erh\"alt Anforderungen, Ziele und Variablen die von Experten bestimmt werden. Danach k\"onnen quantitative Richtwerte berechnet werden auf die sich die Entscheidung zur Auswahl einer bestimmten Hardware st\"utzen kann.