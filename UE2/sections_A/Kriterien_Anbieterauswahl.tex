Es gab viele Versuche alle Kriterien f�r die Auswahl von Ousourcing-Anbietern zu finden, jedoch wurde dieses Problem nocht nicht theoretisch gel�st. Jedoch sollten Firmen alle Faktoren ber�cksichtigen welche den organisatorischen Nutzen beeinflussen k�nnen. Eine Studie welche die wichtigsten Kriterien f�r die Auswahl von Outsourcing-Partnern herausfinden wollte wurde vom "Outsourcing Institute" 2005 durchgef�hrt und lieferte folgende wichtige Kriterien: Preis, Location, flexibler Vertrag, kulturelle �bereinstimmungen, Reputation, bestehende Partnerschaft, Qualit"at, verf"ugbare Ressourcen und die Leistungsf"ahigkeit. Das vorgestellte Entscheidungsmodell st"utzt sich haupts"achlich auf AHP und ELECTREIII Techniken und wird von den oben genannten Faktoren beeinflusst. AHP wird zur Gewichtung der Kriterien verwendet und anschlie"send wird PROMETHEE zum Ranking der Anbieter benutzt.
\cite{4679450}

F"ur immer mehr Unternehmen wird Outsourcing immer wichtiger, deshalb ist es auch sehr wichtig zu wissen wie man die Outsourcing-Partner ausw"ahlt unter der Bedingung die Erwartungen des Unternehmens bestm"oglich zu erf"ullen.
Dieser Prozess der Auswahl ist entscheidend f�r den Wettbewerbsvorteil. Die Anbieterauswahl wird als ein Entscheidungsproblem mit mehreren Kriterien dargestellt.
\cite{5573795}

Cloud Computing wird immer wichtiger f�r Unternehmen und mit der gestiegenen Nachfrage steigt auch das Angebot von verf�gbaren Cloud-Diensten. Es ist jedoch auch wichtig anzumerken, dass sich die Cloud-Dienste wesentlich unterscheiden in Bezug auf Spezifikation, Performance, Preis, etc. Somit ist es nicht einfach f�r die potentiellen Benutzer der Dienste eine gute Entscheidung zu treffen und den richtigen Anbieter auszuw�hlen welcher am besten an die ben�tigten Kriterien zur Verf�gung stellt. Des Weiteren gibt es Bestrebungen im akademischen sowie im unternehmerischen Bereich eine neue Generation von Cloud-Umgebungen zu entwickeln, welche offen und untereinander Kompatibel sind. So w�re ein Szenario m�glich in der ein Benutzer von einem Cloud-Anbieter wechselt und seine Dienste dabei mitnehmen kann, aber es sind noch immer Techniken wichtig um den besten Anbieter dabei auszuw�hlen. Um dieses Problem der Entscheidungsfindung zu bew�ltigen wird das \glqq Multi-Criteria Decision Making\grqq (MCDM)-Modell vorgeschlagen, welches schon in anderen Bereichen seine Effektivit�t bei Entscheidungsproblemen bewiesen hat.
\cite{6468246}