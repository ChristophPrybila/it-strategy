Das komplette Auslagern von Prozessen und Abteilungen wird oftmals von betroffenen Mitarbeitern uns Stakeholdern als negativ empfunden. Arbeitspl"atze und Know-How wird aus der betroffenen Abteilung, Region oder sogar Nation abgezogen. Dies erzeugt Widerstand. "Uberhaupt sind viele Probleme des ''offshore'' Outsourcing sozialer oder personeller Natur. \cite{Mikita2012} Auch wenn der Outsourcing Partner eine gr"o"sere Expertise in IT besitzt, k"onnen die eigenen Mitarbeiter durch ihr Wissen "uber Projekt und Firma effizienter sein. Vor dem Auslagern von Prozessen, muss der Outsourcing Partner auf seine F"ahigkeiten analysiert und ein konkreter Outsourcing plan erstellt werden.

Ein erster Ansatz um einigen negativen Effekten von Outsourcing aus dem Weg zu gehen, ist selektives oder ''smartes'' Sourcing. \cite{Mikita2012} Dies bedeutet, dass z.B. bei IT Projekten ein zentrales Kernteam an Entwicklern behalten wird welche die l"angerfristige Betreuung der Projekte "ubernimmt. Die beschriebenen Probleme verlangen effektives ''Change Management''. Dieser Prozess welcher die Outsourcing Aktivit"aten begleiten soll kann mit verschiedenen Strategien verwirklicht werden.\cite{Mikita2012} 

\begin{itemize}

	\item F"uhrungskraft und und Commitment: Kunden und Outsourcing Unternehmen m"ussen sich "uber die Ziele und den Zweck der gemeinsamen Projekte einig sein. Auch muss klar sein auf welchem Weg die Ziele erreicht werden sollen.
	
	\item Mitarbeiter Effektivit"at: Die Mitarbeiter beider Organisationen m"ussen das Wissen und die Kompetenzen haben, die notwendigen Implementierungsschritte gemeinsam aus zu verhandeln und zu setzen.

 	\item Prozedurale Anpassungen: Die Business-Prozesse des Kunden-Unternehmens m"ussen zusammen mit den beeinflussten Komponenten auf die Ver"anderung vorbereitet und Angepasst werden.

	\item Ver"anderungsakzeptanz: Die verschiedenen Mitarbeiter des Kunden-Unternehmens und die sonstig beeinflussten Stakeholder m"ussen im Ver"anderungsprozess involviert werden. Der Prozess muss verstanden, akzeptiert und unterst"utzt werden.

\end{itemize}

Durch dem Verfolgen dieser verschiedenen Strategien und Konzepte k"onnen das Kunden-Unternehmen und der Outsourcing-Partner gesch"aftlich erfolgreich zusammenarbeiten. 