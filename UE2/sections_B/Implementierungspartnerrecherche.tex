F"ur die Recherche und Bewertung unserer zuk"unfitgen Implementierungspartner beziehen wir uns auf die Erkenntnise vom Outsourcing Institute 2005 (Im Rechercheteil A unter \ref{subsec:impl_partner} auf Seite \pageref{subsec:impl_partner}). Pro Innovationsfeld haben wir uns f"ur jeweils f"unf Technologiepartner bzw. Berater entschieden, welche nach unseren Bewertungs-Kriterien und Charakteristika am geeignetsten sind und das entsprechende Branchen Know-How besitzen. 
\\

Ein wichtiger Punkt ist der Firmenstandort des Umsetzungspartners, dieser sollte, wenn m"oglich, im deutschsprachigem Raum stationiert sein. Im Bereich Mobile Banking musste man aufgrund mangelnder leistungsf"ahiger Partner auch EU-weit bzw. ein amerikanisches Unternehmen evaluieren. 
Weitere wichtige Kriterien f"ur die Rohliste sind generell die Erfahrung und Leistungsf"ahigkeit des Umsetzungspartners in der Bankenbranche. Jeder der Partner sollte schon bei gro"sen und namhaften Banken bzw. Finanzinstitutionen ein Projekt von der Planung und Konzipierung "uber die Implementierung und Umsetzung bis hin zum Rollout erfolgreich realisiert haben.
 
Der externe Partner muss "uber die n"otigen Ressourcen und M"oglichkeiten verf"ugen um auf etwaige Komplikationen bzw. Probleme so rasch wie m"oglich einzugehen und diese zu beseitigen. Ein kundenorientierter Supportcenter, der rund um die Uhr einen kompetenten Ansprechpartner zur Verf"ugung stellt, ist dabei ein unerl"asslicher Aspekt unserer Evaluierung. 
\\

Zu beachten ist auch die Unternehmensgr"o"se und Umsatzst"arke: Ber"ucksichtigt wird ein Implementierungspartner, der sich in einer guten und stabilen finanziellen Situation befindet und auch in Zukunft in der Lage ist erfolgreich zu wirtschaften. Eine "Uberpr"ufung der wirtschaftlichen Leistungsf"ahigkeit und auch des Risikos einer m"oglichen Insolvenz eines Unternehmens ist daher unumg"anglich. Der Gro"steil der angef"uhrten Firmen genie"st weltweit gro"ses Ansehen und hat sich in ihren Gesch"aftsfeldern zu den f"uhrenden Unternehmen bzw. als Marktf"uhrer in ihrem Land etabliert. 

Wie so h"aufig spielt der Preis der Dienstleistung bzw. des Produktes eine wichtige Rolle. In erster Linie sollte der Partner ein vern"unftiges Preis-Leistungsverh"altnis zu markt"ublichen Konditionen gew"ahrleisten. Werden die Anspr"uche und Anforderungen erf"ullt bzw. "ubertroffen und entspricht die Qualit"at den Vorstellungen des Managements, ist man auch bereit in Zukunft weitere Kooperationen einzugehen und anderen Partnern eine positive Empfehlung abzugeben. In diesem Zusammenhang sind flexible Vertr"age seitens des externen Partners sehr w"unschenswert. Der Vertrag sollte Klauseln enthalten, welche ein vorzeitiges Vertragsende erleichtern bzw. bei einer zufriedenen Gesch"aftsbeziehung eine einfache Verl"angerung zu g"unstigeren Konditionen gew"ahrleisten. 