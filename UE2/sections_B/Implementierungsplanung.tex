\newcommand{\workpackage}[6] {
    \paragraph{#1}\mbox{}\\
    \textit{Goal(s):} #2 \\
    \textit{Description:} #3 \\
    \textit{Expected results:} #4 \\
    \textit{Responsible person(s):} #5 \\
    \textit{Dependencies:} #6 
}

Die abgeleiteten Strategien f�r die einzelnen Innovationsfelder werden in drei einzelnen Projekten umgesetzt. Ein Projekt wird jeweils die Strategie eines Innovationsfeldes implementieren. Dies inkludiert Schritte um die notwendige Hard- und Software einzurichten als auch die notwendigen organisatorischen Strukturen im Unternehmen zu integrieren. 

Das Projekt ''mobile banking'' hat wesentliche Abh"angigkeiten zu Projekt ''cloud computing'' und kann deshalb erst nach Abschluss des letzteren begonnen werden. Das Projekt ''contactless payment'' hat keine Abh"angigkeiten zu den anderen beiden Projekten und kann, wenn die Ressourcen vorhanden sind, parallel zu diesen entwickelt werden. Die generellen Projektrisiken werden in einem eigenen Abschnitt beschrieben.\\

\textbf{Generelle Projektorganisation}\\
Jedes Projekt hat einen Koordinator oder Leiter welcher Verantwortlich f"ur die Leitung und Organisation des Projektes ist. Dieser Person obliegt auch die finale Kontrolle und Adaption des Projektfortschrittes. Der gro"steil der offiziellen Kommunikation und das Berichtswesen nach Au"serhalb soll auch "uber diese Position ablaufen. Je nach Umfang sollte es zu dieser Position ein bis zwei Stellvertreter geben. 

Innerhalb jedes Projektes m"ussen je nach aktiven Aufgaben, mehrere Sub-Teams gebildet werden. Jedes Sub-Team ist verantwortlich f"ur eine Aufgabe. Eine Aufgabe kann dabei ein Teil eines Arbeitspaketes sein. Jedes Sub-Team braucht wiederum einen eigenen Team-Kommunikator f"ur die Koordination und Kommunikation nach au"sen.\\

\textbf{Generelle Projektrisiken}\\
Der Projekt Koordinator muss das Risikomanagement w"ahrend des gesamten Projektes "uberwachen. Abgeleitet von den definierten Risiken m"ussen ad"aquate Pr"aventionma"snahmen getroffen werden. Weiteres m"ussen zu jedem Risiko m"ogliche Warnsignale und Reaktionen definiert werden. Die Wahrscheinlichkeit jedes Risikos muss regelm"a"sig "uberwacht und aktualisiert werden.

\begin{itemize}

	\item Der Projektleiter verl"asst das Team

	\item Ein Projektberater oder -experte verl"asst das Team

	\item Ein technischer Mitarbeiter oder Entwickler verl"asst das Team
	
	\item Tempor"arer Ausfall eines Projektmitgliedes

	\item Technische Probleme oder Unzul"anglichkeiten einzelner Technischen Komponenten
	
	\item Mangelnde Qualit"at einzelner Projektartefakte

	\item Ungeplante Kostensteigerung in einzelnen Projektkomponenten

	\item Ungeplante Verl"angerung der Projektdauer

\end{itemize}

Die folgenden Projektpl"ane nehmen keine R"ucksicht auf etwaige Outsourcing Entscheidungen. Sollte ein Projekt nur teilweise durch das eigene Unternehmen umgesetzt werden, dient der beschriebene Projektplan als Grundlage f"ur das Pflichtenheft und den Outsourcing-Vertrag.\\

\textbf{Projektplan ''contactless payment''}\\
\textit{\textbf{Arbeitsschritte}}\\
Die folgenden Liste beschreibt die Sequenz der notwendigen (generellen) Arbeitsschritte zur Durchf"uhrung des Projektes. Weiteres sind verschiedene Meilensteine (MS) an den korrespondieren Stellen eingef"ugt.

\begin{enumerate}

	\item Absch"atzen des zuk"unftigen Kartenbedarfs auf Basis historischer Verkaufsdaten.
	
	\item Definition von Qualit"ats- und Sicherheitsanforderungskatalog	
	
	\item Verhandeln des Kooperationsvertages mit Smartcardlieferanten auf Basis von Kartenbedarf, Qualit"ats- und Sicherheitsanforderungen
	
	\item \textbf{MS\_1:} Kooperationsvertage abgeschlossen
	
	\item Anpassung der bestehenden Smartcard Payment-Applikation auf neues System.
	
	\item Testing von Prototypen gegen die Sicherheitsanforderungen.
	
	\item Testing an den eigenen Bankomaten und an allgemeinen Payment-Terminals
	
	\item \textbf{MS\_2:} Sicherheitsabnahme abgeschlossen
	
	\item Anpassung der bestehenden Kartenausstellungs-Prozesse an den Standorten und in der Zentrale.
	
	\item \textbf{MS\_3:} Integration abgeschlossen

\end{enumerate}

\textit{\textbf{Teambeschreibung}}\\
Zus"atzlich zum Projekt Koordinator werden die folgenden Positionen f"ur das Projekt ben"otigt.

\begin{itemize}
	\item Technische Sicherheitsexperten
	
	\item Marketingexperten
	
	\item Strategische Verhandlungsleiter
	
	\item Technisches Personal welches die Schnittstellen des Unternehmens zur globalen Payment-Infrastruktur verwaltet.
	
	\item Technische Tester
	
\end{itemize}

\textbf{Projektplan ''cloud computing''}\\
\textit{\textbf{Arbeitsschritte}}\\
Die folgenden Liste beschreibt die Sequenz der notwendigen (generellen) Arbeitsschritte zur Durchf"uhrung des Projektes. Weiteres sind verschiedene Meilensteine (MS) an den korrespondieren Stellen eingef"ugt.

\begin{enumerate}

	\item Spezifikation des notwendigen Hard- und Software Bedarfs f"ur die Umstellung
	
	\item Einteilung der Hard- und Software in logische Teilbl"ocke
	
	\item Verhandlung und Vertragsabschluss "uber alle Teilbl"ocke
	
	\item \textbf{MS\_1:} Lieferantenvertr"age abgeschlossen
	
	\item Ankauf und Einrichtung der Shared Service Center Infrastruktur durch ein ausgew"ahltes Kernteam
	
	\item \textbf{MS\_2:} Erstellen der Shared Service Center Infrastruktur
	
	\item �berpr�fung ob Standortspezifischen L"osungen ge�ndert werden m�ssen oder ob weitere services notwendig sind
	\item Planung der Mitarbeiterbewegungen an den Standorten zum Service center

	\item Iterative Einbindung aller Standorte an das Shared Service Center (Beschleunigt mit steigenden Mitarbeiterstab)
	
	\item F"ur jeden Standort:
	
	\begin{enumerate}
		
		\item Umstellung der Prozesse absteigend nach Priorit"at
		
		\item F"ur jeden Prozess:
		
			\begin{enumerate}
			
				\item Anpassung der ben"otigten Standort-Hardware
				
				\item Testlauf jedes angepassten Prozesses
				
			\end{enumerate}
		
		
		\item "Uberpr"ufung der standortspezifischen Prozesse
		
		\item Gegebenenfalls Einrichtung von standortspezifischen Services am Shared Service Center

		\item Verlagern des IT Mitarbeiterstabes in Shared Service Center
	
	\end{enumerate}
	
	\item \textbf{MS\_3:} Standorte umgestellt.	

	\item Entsorgung/Verkauf alter Hardware
	
	\item K�ndigung alter, nicht gebrauchter, Vertr"age

\end{enumerate}

\textit{\textbf{Teambeschreibung}}
Zus"atzlich zum Projekt Koordinator werden die folgenden Positionen f"ur das Projekt ben"otigt.

\begin{itemize}
	\item Technische Netzwerkexperten
	
	\item Technische Serverexperten
	
	\item IT-Strategen des Unternehmens
	
	\item Strategische Verhandlungsleiter
	
	\item Technisches Personal zur Einrichtung der Hard- und Software
	
	\item Personal Strategen des Unternehmens
	
	\item Technische Tester
	
	\item Vertreter der einzelnen Standorte
	
	\item Vertreter der Mitarbeiter
	
\end{itemize}

\textbf{Projektplan ''mobile banking''}\\
\textit{\textbf{Arbeitsschritte}}\\
Die folgenden Liste beschreibt die Sequenz der notwendigen (generellen) Arbeitsschritte zur Durchf"uhrung des Projektes. Weiteres sind verschiedene Meilensteine (MS) an den korrespondieren Stellen eingef"ugt.

\begin{enumerate}

	\item Erhebung der Nutzererwartungen 

	\item Definition der Nutzergruppen (Personas)
	
	\item Definition der gew"unschten Features
	
	\item Spezifikation der Qualit"atseigenschaften
	
	\item Definition von Abnahmetests und Kriterien

	\item Verhandlung und Vertragsabschluss "uber Produkt
	
	\item \textbf{MS\_1:} Produktvertrag abgeschlossen
	
	\item Begleiten des Projektes mit regelm"a"sigen Akzeptanztests
	
	\item Testing von Prototypen gegen die Sicherheitsanforderungen.
	
	\item \textbf{MS\_2:} Sicherheitsabnahme abgeschlossen	
	
	\item Testing durch echte Nutzer (Beta-Test)
	
	\item \textbf{MS\_3:} Nutzerakzeptanz "uberpr"uft
	
	\item Anpassung der bestehenden Payment-Prozesse an den Standorten und in der Zentrale.
	
	\item \textbf{MS\_4:} Integration abgeschlossen

\end{enumerate}

\textit{\textbf{Teambeschreibung}}\\
Zus"atzlich zum Projekt Koordinator werden die folgenden Positionen f"ur das Projekt ben"otigt.

\begin{itemize}
	\item Technische Sicherheitsexperten
	
	\item Marketingexperten
	
	\item Strategische Verhandlungsleiter
	
	\item Technische Usability Experten
	
	\item Technische Tester
	
	\item Repr"asentative Nutzer
	
\end{itemize}

\textbf{Generelle Arbeitspakete}\\
Die folgenden Arbeitspakete fallen in jedem Projekt an. Abgeleitet von den definierten Arbeitsschritten m"ussen weitere Projektspezifische Pakete definiert werden. Horizontale Arbeitspakte  sind ein aufw"andiger aber wichtiger Teil welche "uber die gesamte Projektlaufzeit aktiv sind.

\workpackage{Project Management}
    {Planung und "Uberwachung des Projektes} %goals
    {Leitung, Organisation und Kontrolle des Projektes. Wenn notwendig Adaption des Projektverlaufes.}%description
    {Adaptierung des Planes bei Situations"anderungen.} %expected results
    {Project Manager} %responsible
    {Keine} %dependencies

\workpackage{Unternehmensinternes und -externes Marketing}
    {Steigern der "Offentlichen Wahrnehmung des Projektes} %goals
    {Leistung von interner und externer "Offentlichkeitsarbeit um Kontakt mit internen und externen Stakeholdern zu halten.} %description
    {Steigern der Akzeptanz und des Interesses in das Projekt.} %expected results
    {Kern Entwicklungsteam} %responsible
    {Keine\\} %dependencies