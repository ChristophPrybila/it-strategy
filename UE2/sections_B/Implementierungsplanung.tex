\newcommand{\workpackage}[6] {
    \paragraph{#1}\mbox{}\\
    \textit{Goal(s):} #2 \\
    \textit{Description:} #3 \\
    \textit{Expected results:} #4 \\
    \textit{Responsible person(s):} #5 \\
    \textit{Dependencies:} #6 
}

Die abgeleiteten Strategien f�r die einzelnen Innovationsfelder werden in drei einzelnen Projekten umgesetzt. Ein Projekt wird jeweils die Strategie eines Innovationsfeldes implementieren. Dies inkludiert Schritte um die notwendige Hard- und Software einzurichten als auch die notwendigen organisatorischen Strukturen im Unternehmen zu integrieren. 

Das Projekt ''mobile banking'' hat wesentliche Abh"angigkeiten zu Projekt ''cloud computing'' und kann deshalb erst nach Abschluss des letzteren begonnen werden. Das Projekt ''contactless payment'' hat keine Abh"angigkeiten zu den anderen beiden Projekten und kann, wenn die Ressourcen vorhanden sind, parallel zu diesen entwickelt werden. Die generellen Projektrisiken werden in einem eigenen Abschnitt beschrieben.\\

Die folgenden Projektpl"ane nehmen keine R"ucksicht auf etwaige Outsourcing Entscheidungen. Sollte ein Projekt nur teilweise durch das eigene Unternehmen umgesetzt werden, dient der beschriebene Projektplan als Grundlage f"ur das Pflichtenheft und den Outsourcing-Vertrag.\\

\textbf{Projektplan ''contactless payment''}\\
\textit{\textbf{Arbeitsschritte}}\\
Die folgenden Liste beschreibt die Sequenz der notwendigen (generellen) Arbeitsschritte zur Durchf"uhrung des Projektes zusammen mit deren erwarteten Aufwand in Arbeitsstunden. Weiteres sind verschiedene Meilensteine (MS) an den korrespondieren Stellen eingef"ugt.

\begin{enumerate}

	\item Absch"atzen des zuk"unftigen Kartenbedarfs auf Basis historischer Verkaufsdaten.
	
	\item Definition von Qualit"ats- und Sicherheitsanforderungskatalog	
	
	\item Verhandeln des Kooperationsvertages mit Smartcardlieferanten auf Basis von Kartenbedarf, Qualit"ats- und Sicherheitsanforderungen
	
	\item \textbf{MS\_1:} Kooperationsvertage abgeschlossen
	
	\item Anpassung der bestehenden Smartcard Payment-Applikation auf neues System.
	
	\item Testing von Prototypen gegen die Sicherheitsanforderungen.
	
	\item Testing an den eigenen Bankomaten und an allgemeinen Payment-Terminals
	
	\item \textbf{MS\_2:} Sicherheitsabnahme abgeschlossen
	
	\item Anpassung der bestehenden Kartenausstellungs-Prozesse an den Standorten und in der Zentrale.
	
	\item \textbf{MS\_3:} Integration abgeschlossen

\end{enumerate}

Implementierungsaufwand und die Ressourcen realit"atsnah abzusch"atzen

\textit{\textbf{Teambeschreibung}}\\
\textbf{TBW}\\

\textit{\textbf{Arbeitspakete}}\\
\textbf{TBW}\\

\textit{\textbf{Projektspezifische Risiken}}\\
\textbf{TBW}\\

\textbf{Projektplan ''cloud computing''}\\
\textit{\textbf{Arbeitsschritte}}\\
Die folgenden Liste beschreibt die Sequenz der notwendigen (generellen) Arbeitsschritte zur Durchf"uhrung des Projektes zusammen mit deren erwarteten Aufwand in Arbeitsstunden. Weiteres sind verschiedene Meilensteine (MS) an den korrespondieren Stellen eingef"ugt.

\begin{enumerate}

	\item Spezifikation des notwendigen Hard- und Software Bedarfs f"ur die Umstellung
	
	\item Einteilung der Hard- und Software in logische Teilbl"ocke
	
	\item Verhandlung und Vertragsabschluss "uber alle Teilbl"ocke
	
	\item \textbf{MS\_1:} Lieferantenvertr"age abgeschlossen
	
	\item Ankauf und Einrichtung der Shared Service Center Infrastruktur durch ein ausgew"ahltes Kernteam
	
	\item \textbf{MS\_2:} Erstellen der Shared Service Center Infrastruktur
	
	\item �berpr�fung ob Standortspezifischen L"osungen ge�ndert werden m�ssen oder ob weitere services notwendig sind
	\item Planung der Mitarbeiterbewegungen an den Standorten zum Service center

	\item Iterative Einbindung aller Standorte an das Shared Service Center (Beschleunigt mit steigenden Mitarbeiterstab)
	
	\item F"ur jeden Standort:
	
	\begin{enumerate}
		
		\item Umstellung der Prozesse absteigend nach Priorit"at
		
		\item F"ur jeden Prozess:
		
			\begin{enumerate}
			
				\item Anpassung der ben"otigten Standort-Hardware
				
				\item Testlauf jedes angepassten Prozesses
				
			\end{enumerate}
		
		
		\item "Uberpr"ufung der standortspezifischen Prozesse
		
		\item Gegebenenfalls Einrichtung von standortspezifischen Services am Shared Service Center

		\item Verlagern des IT Mitarbeiterstabes in Shared Service Center
	
	\end{enumerate}
	
	\item \textbf{MS\_3:} Standorte umgestellt.	

	\item Entsorgung/Verkauf alter Hardware
	
	\item K�ndigung alter, nicht gebrauchter, Vertr"age

\end{enumerate}

Implementierungsaufwand und die Ressourcen realit"atsnah abzusch"atzen

\textit{\textbf{Teambeschreibung}}\\
\textbf{TBW}\\

\textit{\textbf{Arbeitspakete}}\\
\textbf{TBW}\\

\textit{\textbf{Spezifische Risiken}}\\
\textbf{TBW}\\

\textbf{Projektplan ''mobile banking''}\\
\textit{\textbf{Arbeitsschritte}}\\
Die folgenden Liste beschreibt die Sequenz der notwendigen (generellen) Arbeitsschritte zur Durchf"uhrung des Projektes zusammen mit deren erwarteten Aufwand in Arbeitsstunden. Weiteres sind verschiedene Meilensteine (MS) an den korrespondieren Stellen eingef"ugt.

\begin{enumerate}

	\item TBW

\end{enumerate}

Implementierungsaufwand und die Ressourcen realit"atsnah abzusch"atzen

\textit{\textbf{Teambeschreibung}}\\
\textbf{TBW}\\

\textit{\textbf{Arbeitspakete}}\\
\textbf{TBW}\\

\textit{\textbf{Spezifische Risiken}}\\
\textbf{TBW}\\

\textbf{Generelle Projektorganisation}\\
 Projekt organisatorisch strukturiert 
\textbf{Generelle Projektrisiken}\\

\textbf{Gesamtaufwand}\\
 und leiten Sie daraus die Gesamtimplementierungsdauer ab.



