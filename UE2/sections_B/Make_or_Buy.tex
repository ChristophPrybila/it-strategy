
TODO

Besonders im Bereich Payment ist ein externe Partner sehr wichtig, weil das eigene aufbauen einer entsprechenden Infrastruktur und entwickeln von Zahlungsm�glichkeiten einen "au"serst zeit- und kostenintensiven Entwicklungsschritt darstellt. \\


Wichtige Kriterien bei der Entscheidungsfindung:
\begin{itemize}
	\item Preis: Der Preis bzw. die Kosten der gebotenen L"osung muss der gebotenen Leistung entsprechen, wobei die bestm"ogliche und wirtschaftlich vertretbare Qualit"at gew"ahrleistet werden muss.  TODO: genauen max. Preis definieren?
	\item Firmensitz: Der Firmensitz bzw. wenigstens qualifizierte Ansprechpartner sollten sich in "Osterreich befinden.
	\item flexibler Vertrag
	\item kulturelle "Ubereinstimmung: Unternehmen mit den selben oder "ahnlichen Arbeitszeiten sind bevorzugt zu behandeln. Da dadruch eine bessere Kommunikation mit dem externen Partner gew"ahrleistet werden kann.
	\item Reputation: Es ist besonders darauf zu achten, ob es in der Vergangenheit Sicherheitsbedenken bei der angebotenen L"osung durch Dritte gegeben hat. Insbesondere sind die Gesch�ftspartner auch einem Screening zu unterziehen in Bezug auf Betrugsf"alle, Betrugsvorw"urfe oder sonstige strafrechtlich relevante Handlungen.
	\item bestehende Partnerschaft: Bestehende Partnerschaften im Banken- und Nichtbankensektor sind zu evaluieren. 
	\item Qualit"at: Die Qualit"at spielt eine wesentliche Rolle bei der Entscheidung insbesondere in Bezug auf die gebotene Sicherheit f"ur die Bank und die Kunden. Nat"urlich muss auch eine entsprechend gute Usability f"ur die Endkunden gegeben sein.
	\item verf"ugbare Ressourcen: Die verf"ugbaren Ressourcen bestimmen im wesentlichen die Leistungsf"ahigkeit des Unternehmens. Insbesondere m"ussen bei den personellen Ressourcen immer Reserven zur Verf"ugung stehen um einen reibungslosen Ablauf bei Problemen und personellen Ausf"allen zu erm"oglichen. Des Weiteren ist die "Uberpr"ufung der finanziellen Lage des Unternehmens wichtig. Eine "Uberpr"ufung ob ein Insolvenzverfahren anh"angig ist, ist unumg"anglich. Im positiven Fall ist dies ein Ausschlusskriterium.
	\item Leistungsf"ahigkeit: Der externe Partner muss in der Lage sein kurzfristig und prompt auf m"oglicherweise aufgetretene Probleme reagieren zu k"onnen. Hier sind besonders die personellen Ressourcen, aber auch die finanziellen Ressourcen zu betrachten.
	
\end{itemize}

Overall Structure - just dumb stuff

Implementierung durch eigene Mitarbeiter(Inhouse) oder von externen Experten(Technologieunternehmen, Berater, Entwickler, etc.


Diskussion im Team
	einzelne Varianten
	Aufw�nde und Kosten in beiden F�llen(Make, Buy)

�bernahme der Informationen und Entscheidungen aus UE1
Erweitern das ganze mit Kriterien:
Preis, Location, flexibler Vertrag, kulturelle �bereinstimmungen, Reputation, bestehende Partnerschaft, Qualit"at, verf"ugbare Ressourcen und die Leistungsf"ahigkeit
Auswahl von eigenen Kriterien und Entscheidungsprozessen. ->
Beschreibung des Prozesses
Wichtig -> Ergebnis muss mit UE1 �bereinstimmen.



Von Ue1-Strategieauswahl:
Zur Realisierung der neuen mobile Banking Plattform bzw. der mo-
bilen Applikation f�r Smartphones und Tablets wird ein externes
IT-Unternehmen beauftragt. Durch die Strategie des Outsourcing die-
ser Teilleistung setzt man weiterhin den Fokus auf das Kerngesch�ft
des Unternehmens. Die Auslagerung bewirkt zudem, dass Kosten
reduziert werden, die f�r zus�tzliches IT-Personal wie Software- und
Webentwicklern entstanden w�ren. Desweiteren erwartet man sich
durch die Kooperation Zugriff auf Know-How bzw. Zugang zu neuen
Technologien, die dazu beitragen l�ngerfristig die Qualit�t zu verbes-
sern.

Mit Hilfe des Ansatzes eines Shared Service Centers soll ein unterneh-
mens�bergreifendes Kompetenzzentrum zur B�ndelung vergleichba-
rer Gesch�ftsprozesse der landesweiten Standorte etabliert werden.
Durch diese Zusammenf�hrung entstehen Synergie-Effekte mit dem
Vorteil, dass einheitliche Schnittstellen und Services f�r alle Standorte
in �sterreich angeboten werden. So ist es m�glich die Flexibilit�t
und Wartbarkeit der Services zu forcieren und durch Skaleneffekte
Betriebskosten einzusparen.

Strategie im Bereich Payment, setzt man auf
die Kooperation mit Markf�hrern aus dem Telekommunikations-
sektor sowie mit f�hrenden Chip- und Endger�teherstellern aus dem
Industriebereich. Mit der Unterst�tzung von einem renommierten
IT-Consultingunternehmen werden in der Branche technologische
Neuerungen und Entwicklungen sowie Trends abgesch�tzt und aus-
gewertet. Oberste Priorit�t hat die soziale Akzeptanz und Verbreitung
des kontaktlosen Zahlens mit Bankomatkarten sowie mobilen Endge-
r�ten in �sterreich

