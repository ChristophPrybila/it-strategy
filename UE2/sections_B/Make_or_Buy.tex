
Die Make or Buy Entscheidung gliedert sich grob in drei Bereiche Mobile Banking, Shared Service Center und Payment Bereich.

Im Bereich Mobile Banking kommt eine Beauftragung eines externen Partners g"unstiger, da unser Unternehmen nicht "uber das notwendige Know-How verf"ugt eine mobile Banking Applikation zu entwickeln. Die Kosten f"ur die Akquisition des Know-Hows durch das einstellen entsprechender Mitarbeiter w"urde mehr Zeit und Geld in Anspruch nehmen. 

Bei dem Shared Service Center muss man zwischen der Hard- und Software unterscheiden. Wobei Letzteres  von Mitarbeitern der Bank(Inhouse) entwickelt werden kann. Da die Mitarbeiter besser mit den abzuwickelnden Gesch"aftsprozessen vertraut sind, als ein externer Partner, der sich erst in die Gesch"aftsprozesse einarbeiten muss. Die Hardware sollte von einem externen Partner zu Verf"ugung gestellt werden welcher im Rahmen seines Vertrags die Infrastruktur(Infrastructure as a Service) f"ur das Shared Service Center zu Verf"ugung stellt.

Besonders im Bereich Payment ist ein externe Partner sehr wichtig, weil das eigene aufbauen einer entsprechenden Infrastruktur und entwickeln von Zahlungsm�glichkeiten einen "au"serst zeit- und kostenintensiven Entwicklungsschritt darstellt. \\


\textbf{Details zu Kosten Make}

\bigskip
\begin{minipage}{\linewidth}
\centering
\begin{tabular}{|l|p{5cm}|}
\hline
\textbf{Bereich} & \textbf{Aufw"ande}  \\
\hline
Mobile Banking &  Entwickeln der Applikation, Sicherheits"uberpr"ufung durch die Bank \\
\hline
Shared Service Center - Infrastruktur & Aufbauen einer Serverfarm(Immobilienkauf, Serverkomponenten), 
einstellen entsprechender Mitarbeiter zur Instandsetzung und Instandhaltung   \\
\hline
Shared Service Center - Software & Entwickeln der Bankgesch"aftssoftware, Sicherheits"uberprufung \\ 
\hline
Payment & Entwickeln einer Paymentl"osung, Werbung f"ur neue Paymentl"osung \\ 
\hline
\end{tabular} 
\par
\bigskip
Details zu den Kosten der Make-Entscheidung
\end{minipage} \\

\textbf{Details zu Kosten Buy}

\bigskip
\begin{minipage}{\linewidth}
\centering
\begin{tabular}{|l|p{5cm}|}
\hline
\textbf{Bereich} & \textbf{Aufw"ande}  \\
\hline
Mobile Banking & Suche des externen Partners, Auftragsvergabe, Sicherheits"uberprufung und Abnahme der Applikation \\
\hline
Shared Service Center - Infrastruktur & Suche des externen Partners, Auftragsvergabe \\
\hline
Shared Service Center - Software & Suche des externen Partners, Auftragsvergabe, Details der Gesch"aftsprozesse kommunizieren, st"andige Beratung w"ahrend der Entwicklung, eingehende Sicherheits"uberpr"ufung und Tests, Abnahme  \\ 
\hline
Payment & Suche des externen Partners, Auftragsvergabe, Abnahme  \\ 
\hline

\end{tabular} 
\par
\bigskip
Details zu den Kosten der Buy-Entscheidung
\end{minipage} \\

\textbf{Conclusio}

Deshalb wird empfohlen besonders in den Bereichen Payment und mobile Banking einen externen Partner mit der Entwicklung zu beauftragen.  Des Weiteren wird es empfohlen das Shared Service Center selbst zu betreiben, jedoch bei der Hardware(in Form von IaaS) auf einen externen Partner zu setzen, um die hohen Sicherheitsstandards leichter kontrollieren zu k"onnen und die Kontrolle "uber wichtige Gesch"aftsprozesse und Know-How der Bank zu erhalten.

\textbf{Wichtige Kriterien bei der Entscheidungsfindung}
\begin{itemize}
	\item Preis: Der Preis bzw. die Kosten der gebotenen L"osung muss der gebotenen Leistung entsprechen, wobei die bestm"ogliche und wirtschaftlich vertretbare Qualit"at gew"ahrleistet werden muss. 
	\item Firmensitz: Der Firmensitz bzw. wenigstens qualifizierte Ansprechpartner sollten sich in "Osterreich befinden.
	\item flexibler Vertrag: Der Vertrag sollte Klauseln enthalten die einen vorzeitiges Vertragsende erleichtern bzw. bei einer guten Gesch"aftsbeziehung eine einfache Verl"angerung. Des Weiteren sollten auch die Leistungen leicht und unb"urokratisch den Umweltbedingungen angepasst werden k"onnen.
	\item kulturelle "Ubereinstimmung: Unternehmen mit den selben oder "ahnlichen Arbeitszeiten sind bevorzugt zu behandeln. Da dadruch eine bessere Kommunikation mit dem externen Partner gew"ahrleistet werden kann.
	\item Reputation: Es ist besonders darauf zu achten, ob es in der Vergangenheit Sicherheitsbedenken bei der angebotenen L"osung durch Dritte gegeben hat. Insbesondere sind die Gesch�ftspartner auch einem Screening zu unterziehen in Bezug auf Betrugsf"alle, Betrugsvorw"urfe oder sonstige strafrechtlich relevante Handlungen.
	\item bestehende Partnerschaft: Bestehende Partnerschaften im Banken- und Nichtbankensektor sind zu evaluieren. 
	\item Qualit"at: Die Qualit"at spielt eine wesentliche Rolle bei der Entscheidung insbesondere in Bezug auf die gebotene Sicherheit f"ur die Bank und die Kunden. Nat"urlich muss auch eine entsprechend gute Usability f"ur die Endkunden gegeben sein.
	\item verf"ugbare Ressourcen: Die verf"ugbaren Ressourcen bestimmen im wesentlichen die Leistungsf"ahigkeit des Unternehmens. Insbesondere m"ussen bei den personellen Ressourcen immer Reserven zur Verf"ugung stehen um einen reibungslosen Ablauf bei Problemen und personellen Ausf"allen zu erm"oglichen. Des Weiteren ist die "Uberpr"ufung der finanziellen Lage des Unternehmens wichtig. Eine "Uberpr"ufung ob ein Insolvenzverfahren anh"angig ist, ist unumg"anglich. Im positiven Fall ist dies ein Ausschlusskriterium.
	\item Leistungsf"ahigkeit: Der externe Partner muss in der Lage sein kurzfristig und prompt auf m"oglicherweise aufgetretene Probleme reagieren zu k"onnen. Hier sind besonders die personellen Ressourcen, aber auch die finanziellen Ressourcen zu betrachten.
	
\end{itemize}
