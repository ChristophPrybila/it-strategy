\textbf{Anbieter-Shortlist f"ur Cloud Server \& Software}\\
F"ur das Innovationsfeld Cloud Server \& Software befinden sich folgende Anbieter der vorher erstellten Longlist in der Top 10 Shortlist:

\begin{itemize}
	\item Atos
	\item IBM
	\item HP
	\item Teradata
	\item Oracle
	\item SAP
	\item EMC
	\item Microsoft
	\item VMware
	\item NetApp
\end{itemize}

F"ur die Auswahl der pr"aferierten Anbieter war es vor allem wichtig, dass Hardware und Software von einem Anbieter stammen um die Kompatibilit"at zu gew"ahrleisten. Weiters wurde besonderes Augenmerk darauf gelegt, ob der Anbieter einerseits L"osungen f"ur eine Private Cloud (Kernprozesse) anbietet und andererseits Supportprozesse auf die IT-Infrastruktur des jeweiligen Unternehmens ausgelagert werden k"onnen. Zu guter Letzt wurde auch die Erfahrung in der Bankenbranche und erfolgreich abgeschlossene Projekte in den Entscheidungsprozess mit einbezogen.\\

\textbf{Pr"aferierte Hardware L"osung}

\begin{description}

	\item[Atos] F"ur das allgemeine Unternehmensnetzwerk bietet Atos vorgefertigte ''Network and Communications'' L"osungen an. Die private Cloud f"ur die Kernprozesse kann durch die Einbindung einer ''	Managed Infrastructure'' L"osung erstellt werden. Schlussendlich bietet Atos auch noch Beratung bei Outsourcing Entscheidungen an, welche f"ur die Supportprozessen in Frage kommen.
	
	\item[IBM] Dieser Anbieter bietet eine Vielzahl von Cloud-L�sungen mit verschiedensten Anpassungen an. Ob ein Basispaket f"ur die Kernprozesse ausreicht, oder ob weitere Pakete ben"otigt werden muss in einer detaillierteren Analyse evaluiert werden. F"ur das optionale Auslagern von Supportprozessen bietet IBM ebenfalls Beratung zu IT-Outsourcing an.
	
	\item[HP] Cloud-Services sind bei diesem Anbieter unter dem Synonym ''Converged Cloud'' zusammengefasst. Dabei wird sowohl der Eigenbetrieb ("Offentliche Cloud) als auch der externe Betrieb der Cloud durch HP angeboten. Diese "Offentliche Cloud auf Open-Source Basis kann f"ur die Kernprozesse im Unternehmen eingesetzt werden. Supportprozesse k"onnen optional in die HP Cloud ausgelagert werden.
	
	
\end{description}
