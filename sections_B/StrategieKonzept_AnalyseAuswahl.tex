\textbf{Strategische Analyse}\\

Bazinga Bank ist ein im Jahr 2010 gegr"undetes "osterreichiches Gro"sunternehmen in der Bankenbranche mit mehreren Standorten und mittlerweile mehr als 700 Mitarbeitern in "Osterreich. Aufgrund der immer zunehmenden Digitalisierung im Informationszeitalter hat sich XYZ als Ziel gesetzt innerhalb der n"achsten 3 Jahre den Prozess der digitalen Transformation hin zu einem zeitgem"a"sen und innovativen Unternehmen zu forcieren. Speziell IT-Infrastruktur und IT-Strategie des Unternehmens sollen analysiert und neu konzipiert werden um die von der Gesch"aftsf"uhrung definierten Gesch"aftsziele zu erreichen. In diesem Zusammenhang werden laufend Trends sowie technologische Entwicklungen und Neuerungen am Markt in Kooperation mit einem renommierten IT-Beratungsunternehmen beobachtet und evaluiert. Diese Zusammenarbeit dient als Unterst"utzungsgrundlage f"ur wichtige Entscheidungen in Hinblick auf die IT-Strategie des Unternehmens. Das standardm"a"sige Internetbanking wird mit der mobile banking Komponente erg"anzt. Dem Kunden sollen externe Services intuitiv und einfach auch au"serhalb der Filialen und "Offnungszeiten direkt auf dem mobilen Endger"at zur Verf"ugung gestellt werden, nach dem Prinzip \glqq service anywhere and anytime\grqq. Durch die Forcierung von Synergieeffekten sollen Kerngesch"aft, interne und externe Businessprozesse sowie Know-How geb"undelt werden um eine unternehmensweite Vereinheitlichung sicher zustellen und Wettbewerbsvorteile zu erzielen. Im Finanz- und Paymentbereich wird die Innovationsf"uhrerschaft durch laufende Erforschung und Bewertung neuester Trends und technologischer Entwicklungen von Bezahlsystemen angestrebt. Eine Teilnahme bzw. Mitarbeit an relevanten Industrie-Konsortien und Konferenzen im Finanz- und Paymentbereich ist hier Vorraussetzung zum Erreichen des Ziels. Auch in puncto Sicherheit wird nichts dem Zufall "uberlassen, weil  sie ein unerl"assliches Qualit"atsmerkmal f"ur jedes Unternehmen darstellen sollte. Dementsprechend versteht man unter Sicherheit einen iterativen und inkrementellen Prozess, welcher im Einklang mit der Unternehmensphilosophie steht. Durch "Ubernahme von Security-Standards bzw. Zertifizierungen sowie regelm"a"sigen Qualit"atskontrollen soll das Sicherheitsniveau m"oglichst hoch gehalten werden um die Vertrauensbildung bei Kunden und Mitarbeiten zu allen Services und Produkten zu st"arken. \\

\textbf{Strategieauswahl}\\

\begin{itemize}

	\item Zur Realisierung der neuen mobile Banking Plattform bzw. der mobilen Applikation f"ur Smartphones und Tablets wird ein externes IT-Unternehmen beauftragt. Durch die Strategie des Outsourcing dieser Teilleistung setzt man weiterhin den Fokus auf das Kerngesch"aft des Unternehmens. Die Auslagerung bewirkt zudem, dass Kosten reduziert werden, die f"ur zus"atzliches IT-Personal wie Software- und Webentwicklern entstanden w"aren. Desweiteren erwartet man sich durch die Kooperation Zugriff auf Know-How bzw. Zugang zu neuen Technologien, die dazu beitragen l"angerfristig die Qualit"at zu verbessern. \\
		
	
	\item Ausn"utzung von Synergie-Effekten auf Produkt- und Business-Prozess Ebene. Anbieten von einheitlichen Schnittstellen und Services f"ur alle Standorte. Forcieren der Flexibili"at und Wartbarkeit der Services und somit der dar"uberliegenden Services. 

	\item Mit Hilfe des Ansatzes eines Shared Service Centers soll ein unternehmens"ubergreifendes Kompetenzzentrum zur B"undelung vergleichbarer Gesch"aftsprozesse der landesweiten Standorte etabliert werden. Durch diese Zusammenf"uhrung entstehen Synergie-Effekte mit dem Vorteil, dass einheitliche Schnittstellen und Services f"ur alle Standorte in "Osterreich angeboten werden. So ist es m"oglich die Flexibilit"at und Wartbarkeit der Services zu forcieren und durch Skaleneffekte Betriebskosten einzusparen.\\

	\item Zur Forcierung der Strategie im Bereich Payment, setzt man auf die Kooperation mit Markf"uhrern aus dem Telekommunikationssektor sowie mit f"uhrenden Chip- und Endger"ateherstellern aus dem Industriebereich. Mit der Unterst"utzung von einem renommierten IT-Consultingunternehmen werden in der Branche technologische Neuerungen und Entwicklungen sowie Trends abgesch"atzt und ausgewertet. Oberste Priorit"at hat die soziale Akzeptanz und Verbreitung des kontaktlosen Zahlens mit Bankomatkarten sowie mobilen Endger"aten in "Osterreich.\\


	\item Sicherheit bildet das Fundament um Vertrauen bei Kunden und Mitarbeitern in die Produkte und Services des Unternehmens aufzubauen. Das hohe Sicherheitsniveau wird gew"ahrleistet in dem verschiedene Security-Standards und Normen "ubernomen werden und die Zertifizierung durch die OCG("Oesterreichische Computer Gesellschaft) Zertifizierungstelle erfolgt.
	
\end{itemize}