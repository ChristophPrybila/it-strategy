\textbf{Anbieter-Shortlist f"ur Smartcard \& Embedded Applikationen}\\
F"ur das Innovationsfeld Contactless Payment befinden sich folgende Anbieter in der vorher erstellten Longlist in der Top 10 Shortlist:

\begin{itemize}
	\item Maestro
	\item paylife
	\item Austriacard
	\item TNS Pay
	\item CoreFire
	\item FIME
	\item Accenture LLP
	\item Datacard Deutschland Group
	\item 3M Cogent Systems
	\item Clear 2 Pay
\end{itemize}

Ein wichtiges Kriterium f"ur die engere Auswahl ist, wie auch bei den anderen Innovationsfeldern, generell die Erfahrung der Anbieter in der Bankenbranche. Des Weiteren ist ebenfalls zu beachten: die Unternehmensgr"o"se und Umsatzst"arke: Ber"ucksichtigt wird ein Anbieter, der sich in einer guten und stabilen finanziellen Situation befindet und auch in Zukunft in der Lage ist erfolgreich zu wirtschaften. Jedoch ist ein weiteres wichtiges Kriterium die Sicherheit der Contactless Payment-Systeme um unautorisierten Zugriff durch Dritte zu vermeiden. \\

\textbf{Pr"aferierte Partnerschafts L"osung}

\begin{description}
	
	\item[paylife] Dies ist der Anbieter f"ur die Ausstellung von international kompatiblen und kontaktlosen Debitkarten. Als m"ogliche Akzeptanzvertr�ge werden V-Pay und Maestro angeboten. Weiteres ist paylife auch ein gro"ser Anbieter von Bezahlterminals, was bei der Entwicklung der elektronischen Geldb"orse unbedingt kompatibel sein muss.

	\item[Austriacard] Dies ist der Anbieter f"ur SmartCard Entwicklung in "Osterreich. Er bietet die gr"o"stm"ogliche Compliance zu internationalen Payment- und Sicherheitsstandards. Er fungiert oftmals als Lieferant f"ur alle anderen Anbieter von Bankomatkarten-L"osungen. Jede Partnerschaft "uber kontaktlose Bankomatkarten sollte mit Hinblick auf diese Firma abgeschlossen werden. Als Teil der ''Wallet Initiative'' ist Austriacard auch ein geeigneter Partner f"ur die Entwicklung und den Einsatz von elektronischen Geldb"orsen.

\end{description}