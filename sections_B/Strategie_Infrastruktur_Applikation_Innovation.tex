\textbf{Infrastrukturstrategie}\\
Um bei Verhandlungen "uber Investitionen und Wartungen von Infrastruktur gegen"uber den Anbietern eine gr"o"sere Marktmacht zu besitzen, ist das grunds"atzliche Ziel die im Unternehmen eingesetzten Komponenten zu vereinheitlichen. Das zentrale Management f"ur die generell eingesetzte Hardware wird vom \textcolor{red}{zentralen Service Center} durchgef"uhrt. Weiteres definiert das \textcolor{red}{zentralen Service Center} die Vorgaben f"ur alle lokal-spezifisch eingesetzten Komponenten an den Standorten.

Der Infrastruktur-Bedarf von generelle Kernprozessen, welche f"ur alle Standorte gleich sind, kann geb"undelt und einheitlich verwaltet werden. 

Der Infrastruktur-Bedarf von generelle Supportprozessen, welche f"ur alle Standorte gleich sind, kann geb"undelt und einheitlich verwaltet werden. 

Der Infrastruktur-Bedarf von standort-spezifischen Prozessen kann nicht geb"undelt und vereinheitlicht werden. Die speziell ben"otigten Komponenten m"ussen extra angeschafft und verhandelt werden.

\begin{description}

	\item[Hardware \& Betriebssysteme] F"ur die verschiedenen (generellen) Aufgabenprofile innerhalb des Unternehmens (Verwaltung, Entwicklung, Kundenbetreuung, etc.) werden einheitliche Workstations und Server definiert welche auf die jeweiligen Anforderungen zugeschnitten sind. Definiert werden die Hardware-technischen Anforderungen und das, in Hinsicht auf die verwendeten Applikationen und Services notwendige, Betriebsystem. Bei der Wahl des Betriebsystems ist bei verschiedenen M"oglichkeiten, der kosteng"unstigeren oder open-source Variante der Vorzug zu geben.
	
	\item[Interne Netzwerke] Die Administration von standortinternen Netzwerken obliegt jeweils den lokalen Administratoren, wobei jedoch diese nach den zentralen Vorgaben zu erfolgen hat. Die verwendeten Router und Switches sowie die verlegten Kabel werden zentral vorgegeben. F"ur jeden Standort m"ussen lokale Server die lokale User-, Daten- und Kommunikationsverwaltung "ubernehmen. F"ur jeden eingesetzten Server m"ussen die nach Aufgabenprofil definierten Vorgaben eingehalten werden. Alle standortinternen User und Daten sind auf einem Server abgelegt. Weiteres soll ein weiterer Server die Schnittstelle zu externen Netzwerken bieten. Es obliegt diesem Server, f"ur die lokalen Workstations die Schnittstelle zu den notwendigen zentralen Applikationen und Funktionalit"aten bereitzustellen. Lokal verwaltetet Applikationen und Funktionalit"aten m"ussen "uber einen weiteren lokalen Server angeboten werden.
	
	\item[Externe Netzwerke] Bei externen Netzwerken wird  das geographisch verteilte Unternehmensnetzwerk und das Internet selbst unterschieden. Beide Netzwerke werden "uber die Anschl"usse von Internet-Service-Provider (ISP) betrieben. Alle Standorte sollen ihren Anschluss vom selben ISP zu verf"ugung gestellt bekommen. Es m"ussen die erwarteten Netzwerk- belastungen und -anforderungen f"ur die einzelnen Standorte ermittelt werden und generelle Leistungsklassen definiert werden. Jedem Standort wird eine Leistungsklasse zugeordnet. Der gew"ahlte ISP muss in der Lage sein alle Leistungsklassen f"ur alle Standorte zu erf"ullen. Mit dem ISP m"ussen ad"aquate "Uberpr"ufungen und Messmethoden vereinbart werden. Das Unternehmensnetzwerk soll als Virtual-Private-Network (VPN) abgebildet und betrieben werden. Die notwendige Technologie hierzu soll ebenfalls von einem einzigen Anbieter kommen, muss in diesem Fall aber von den eigenen IT-Mitarbeitern zu administrieren sein.

\end{description}

\textbf{Applikationsstrategie}\\
TBW Christoph\\\\



\textbf{Innovationsstrategie}\\
TBW Christoph\\\\