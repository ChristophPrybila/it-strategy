\textbf{Infrastrukturstrategie}\\
Um bei Verhandlungen "uber Investitionen und Wartungen von Infrastruktur gegen"uber den Anbietern eine gr"o"sere Marktmacht zu besitzen, ist das grunds"atzliche Ziel die im Unternehmen eingesetzten Komponenten zu vereinheitlichen. Das zentrale Management f"ur die generell eingesetzte Hardware wird vom \textcolor{red}{zentralen Service Center} durchgef"uhrt. Weiteres definiert das \textcolor{red}{zentralen Service Center} die Vorgaben f"ur alle lokal-spezifisch eingesetzten Komponenten an den Standorten.

\begin{itemize}

	\item Der Infrastruktur-Bedarf von \textbf{generelle Kernprozessen}, welche f"ur alle Standorte gleich sind, kann geb"undelt und einheitlich verwaltet werden. 

	\item Der Infrastruktur-Bedarf von \textbf{generelle Supportprozessen}, welche f"ur alle Standorte gleich sind, kann geb"undelt und einheitlich verwaltet werden. 

	\item Der Infrastruktur-Bedarf von \textbf{standort-spezifischen Prozessen} kann nicht geb"undelt und vereinheitlicht werden. Die speziell ben"otigten Komponenten m"ussen extra angeschafft und verhandelt werden.

\end{itemize}

\begin{description}

	\item[Hardware \& Betriebssysteme] F"ur die verschiedenen (generellen) Aufgabenprofile innerhalb des Unternehmens (Verwaltung, Entwicklung, Kundenbetreuung, etc.) werden einheitliche Workstations und Server definiert welche auf die jeweiligen Anforderungen zugeschnitten sind. Definiert werden die Hardware-technischen Anforderungen und das, in Hinsicht auf die verwendeten Applikationen und Services passende, Betriebssystem. Bei der Wahl des Betriebssystems ist bei verschiedenen M"oglichkeiten, der kosteng"unstigeren oder (nach m"oglichkeit open-source) Variante der Vorzug zu geben.
	
	\item[Interne Netzwerke] Die Administration von standortinternen Netzwerken obliegt jeweils den lokalen Administratoren, wobei jedoch diese nach den zentralen Vorgaben zu erfolgen hat. Die zu verwendenden Router und Switches sowie Netwerkkabel werden zentral vorgegeben. Das Netzwerk darf wegen der zu gro"sen Angriffsfl"ache nicht "uber WLan betrieben werden. F"ur jeden Standort m"ussen lokale Server vorhanden sein, welche lokale die User-, Daten- und Kommunikationsverwaltung "ubernehmen. F"ur jeden eingesetzten Server m"ussen die nach Aufgabenprofil definierten Vorgaben eingehalten werden. Alle standortinternen User und Daten sind auf einem lokalen Server abgelegt. Weiteres soll ein weiterer Server die Schnittstelle zu externen Netzwerken bieten. Es obliegt diesem Server, f"ur die lokalen Workstations die Schnittstelle zu den notwendigen zentralen Applikationen und Funktionalit"aten bereitzustellen. Lokal verwaltetet Applikationen und Funktionalit"aten m"ussen "uber einen weiteren lokalen Server angeboten werden.
	
	\item[Externe Netzwerke] Bei externen Netzwerken wird zwischen dem geographisch verteilten Unternehmensnetzwerk und dem Internet selbst unterschieden. Beide Netzwerke werden "uber die Anschl"usse von Internet-Service-Provider (ISP) betrieben. Alle Standorte sollen ihren Anschluss vom selben ISP zu verf"ugung gestellt bekommen. Es m"ussen die erwarteten Netzwerkbelastungen und Netzwerkanforderungen f"ur die einzelnen Standorte ermittelt werden und generelle Leistungsklassen definiert werden. Jedem Standort wird eine Leistungsklasse zugeordnet. Der gew"ahlte ISP muss in der Lage sein alle Leistungsklassen f"ur alle Standorte zu erf"ullen. Mit dem ISP m"ussen ad"aquate "Uberpr"ufungen und Messmethoden vereinbart werden. All diese Informationen werden in geeigneten SLAs definiert und vereinbart. Das geographisch verteilte Unternehmensnetzwerk soll als Virtual-Private-Network (VPN) abgebildet und betrieben werden. Die notwendige Technologie hierzu soll ebenfalls von einem einzigen Anbieter kommen, muss in diesem Fall aber von den eigenen IT-Mitarbeitern zu administrieren sein.

\end{description}

\textbf{Applikationsstrategie}\\

\begin{description}
	
	\item[Applikationen f"ur generelle Kernprozesse] Um einheitliche Prozesse und Produkte an allen Standorten zu gew"ahrleisten werden die Applikationen, welche notwendig f"ur Kernprozesse sind, in der unternehmenseigenen Cloud betrieben. Die Workstations agieren somit nur als Thin-Clients, die lokal betriebenen Applikationen agieren nur als Schnittstelle welche den Input der Nutzer in die Cloud einspei"en. Je nach Auslastung sollen verschiedene Instanzen einer Applikation zu verf"ugung stehen welche normalerweise im \textcolor{red}{zentralen Service Center} betrieben werden. Bei starker Auslastung kann eine solche Instanz aber auch auf einen Server des jeweiligen Standortes verschoben werden. Wichtig ist, dass die verwendete Datenquelle aller Instanzen immer zentral bleibt um die notwendige Konsistenz zu gew"ahrleisten. Hier muss bei gro"sen Datenbedarf, auf intelligente Cacheing-Strategien und ausreichende Bandbreite im Netzwerk geachtet werden. Sollte dies nicht m"oglich sein muss eventuelle Konsistenz ausreichen. Das bedeutet, dass der Datenbestand au"serhalb der Gesch"aftszeiten synchronisiert werden muss. Die Entwicklung und Wartung dieser Applikationen soll im Unternehmen stattfinden. Um Wartbarkeit und Interoperabilit"at zu gew"ahrleisten sollen die Applikationen als Services einer Software-Oriented-Architecture (SOA) innerhalb eines Enterprise-Service-Bus (ESB) betrieben werden.  
	
	\item[Applikationen f"ur Supportprozesse] Applikationen welche in diese Kategorie fallen sollen ebenfalls einheitlich und zentral betrieben werden. Um Kapazit"aten zu sparen kann die Entwicklung und Wartung dieser Applikationen gegebenenfalls auch von einer externen Firma durchgef"uhrt werden. Notwendig hierbei sind jedoch spezifische Sicherheits- und Kompatibilit"atsdefinitionen. Applikationen bei welchen eventuelle Konsistenz ausreicht sollen auf lokalen Servern betrieben werden. 
	
	\item[Applikationen f"ur standortspezifische Kernprozesse] Applikationen f"ur Standortspezifische Kernprozesse m"ussen lokal entwickelt und betrieben werden. Hierbei sind zentrale Architektur- und Schnittstellenvorgaben einzuhalten um die notwendige Sicherheit und Kompatibilit"at zu gew"ahrleisten. Solche kostenintensive Applikationen und Produkte mu"ssen sich in regelem"a"sigen Abst"anden einer Wirtschaftlichkeitspr"ufung unterziehen. Bei erfolgreichen Produkten wird eine "ubernahme ins zentrale Portfolio gepr"uft. 

	\item[Office \& Email Applikationen] Applikationen f"ur Office und Kundenkontakt sollen einheitlich vom \textcolor{red}{zentralen Service Center} definiert werden. Notwendige Lizenzen f"ur Wartung und Updates sollen f"ur alle Standorte zusammen gekauft werden.

\end{description}

\textbf{Innovationsstrategie}\\
TBW Christoph\\\\