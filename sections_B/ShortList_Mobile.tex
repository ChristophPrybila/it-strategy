\textbf{Anbieter-Shortlist f"ur mobile Applikationen}\\
F"ur das Innovationsfeld Mobile Banking befinden sich folgende Anbieter in der vorher erstellten Longlist in der Top 10 Shortlist:

\begin{itemize}
	\item FIS
	\item Monitise Group Limited
	\item Fiserve Inc.
	\item Infosys
	\item Digital Insight Corp.
	\item Jack Henry
	\item Kony Inc.
	\item Sybase EDV-System GmbH.
	\item ACI Worldwide Germany
	\item Misys
\end{itemize}

Ein wichtiges Kriterium f"ur die engere Auswahl ist generell die Erfahrung der Anbieter in der Bankenbranche. Die Anbieter sollten schon bei gro"sen und namhaften Banken bzw. Finanzinstitutionen die Konzipierung und Implementierung von Mobile	 Banking Infrastrukturen und Applikationen erfolgreich durchgef"uhrt haben. Zu beachten ist auch die Unternehmensgr"o"se und Umsatzst"arke: Ber"ucksichtigt wird ein Anbieter, der sich in einer guten und stabilen finanziellen Situation befindet und auch in Zukunft in der Lage ist erfolgreich zu wirtschaften. Zudem sollte die mobile Applikation auf g"angigen Plattformen laufen, um viele unterschiedliche Endger"ate zu unterst"utzen. Ein weiteres wichtiges Kriterium ist, dass der Anbieter auf Sicherheitsebene 2-Faktor bzw. Multi-Faktor Authentifizierung zur Verf"ugung stellt um den hohen Sicherheitsanforderungen gerecht zu werden.\\

\textbf{Pr"aferierte Software L"osungen}
\begin{description}

	\item[FIS - Commercial eBanking]   TBW
	
	\item[Fiserve Inc. - Consumer Online Banking]   TBW Adaption und Anpassung des Produktions 
	
	\item[Monitise Group Limited - Bank Anywhere]   TBW
	
\end{description}