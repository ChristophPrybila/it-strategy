\textbf{Strategieumsetzung}\\

Chris: \textcolor{red}{Alte Texte von den Zielen passen hier besser. Sortieren nach den jeweiligen IT-Ziel.\\ M"ussen weiteres noch in einzelne Teilschritte umformuliert werden.\\
Einbeziehung von Strategien welche in der letzten Section definiert wurden (z.b. Shared Service Center oder Outsourcing
		Unternehmensweiten Kernbereichen, Kernprozessen und Kern-Knowhow
		unternehmensweiten Supportprozessen
		standortspezifischen Kernbereichen, Kernprozessen und Kern-Knowhow}

	\begin{itemize}
		\item Die Einf"uhrung des mobilen Banking-Portals wird unterst"utzt durch regelm"a"sige usability Tests und Studien um das Produkt m"oglichst kundenfreundlich zu gestalten.
		\item Ein externes Call Center, welches zu den Gesch"aftstzeiten erreichbar ist soll aktiv die Nutzung der mobilen Plattform anpreisen und dem Kunden bei etwaigen Fragen bez"uglich der Bedienung behilflich sein. Bei Bank bzw. Finanzdienstleistungs bezogenen Fragen wird mit einem bankinternen Berater verbunden.

	\item Einf"uhrung eines zentralen Cloud-Systems welches die notwendigen internen Services ''on demand'' allen Standorten zu Verf"ugung stellt. 
	\begin{itemize}
		\item Durch smarte und skalierbare IT-Infrastruktur, wird eine nahezu 100\%ige Verf"ugbarkeit sichergestellt. Ein t�glich geplantes Wartungsfenster zwischen 4 und 5 Uhr in der fr"uh wird nicht in die Berechnung der Verf"ugbarkeit miteinbezogen.
	\item Die Hardware wird mit Unterst"utzung einer externen, vertrauensw"urdigen Firma betrieben, um m"ogliche Serverausf"alle so rasch wie m"oglich zu beheben.
	\item Des Weiteren werden Vertr"age mit Telekommunikationsanbietern abgeschlossen um die ben"otigte Verf"ugbarkeit und Bandbreite des Netzwerks zu garantieren.
	\end{itemize}
		
	\item Alle neu ausgestellten Bankomatkarten werden mit der NFC-Technologie ausger"ustet um innerhalb der n"achsten 2 Jahre alle Kunden mit dieser Technologie ausgestattet zu haben.
	\item Vertr"age mit Chipkartenherstellern sollen sicherstellen, dass embedded Bankomatkarten in einer SIM-Karte bereitgestellt werden k�nnen(Es ist anzunehmen, dass innerhalb der n"achsten zwei Jahre 80\% der neuen Mobiltelefone mit einer NFC-Technologie ausgestattet werden), um ein mobiles bezahlen mit dem Handy zu erm"oglichen.
\\
	
	\item IT-Sicherheit
\begin{itemize}
\item Sicherheit ist ein durchgehender Prozess w"ahrend der Software-Entwicklung. Die Sicherheits"uberpr"ufung der einzelnen Komponenten allein (Modultest) und interagierender Komponenten untereinander (Integrationstest). Bei der Entwicklung wird anfangs durch die Erstellung der Tests ein Mehraufwand entstehen, jedoch wird dieser durch die Automatisierung der Tests und die dadurch fr"uhere Fehlererkennung  weitgehend kompensiert. Durch dieses Verfahren im Entwicklungsprozess wird die Reliability auf ein maximales Ma"s angehoben.

\item Nach der Fertigstellung der Software-Komponenten werden regelm�"sige Security-Audits abgehalten und Penetration-Tests durchgef"uhrt. Bei den monatlich abgehaltenen Security-Audits werden alle Mitarbeiter eingebunden und m"ogliche Schwachstellen aufgezeigt. Hierbei wird nicht nur die Software betrachtet sondern auch allgemeine Vorgehensweisen des Unternehmens und Mitarbeiter, wie zum Beispiel das Betreten eines abgesicherten Bereichs(z.b. Serverraum). Die monatlichen Audits dauern 1-2 Stunden und sind somit mit einem geringen Mehraufwand verbunden. Systematische Penetration-Tests sind mit deutlichem Mehraufwand verbunden und werden deshalb in eine eigens geschulte Abteilung ausgelagert.

\item Grunds"atzlich kommt immer das vier Augen Prinzip zur Anwendung, wenn "Anderungen an der IT-Infrastruktur notwendig sind. Dies dient zum einem der Senkung von Bedienfehlern, andererseits vermeidet es auch den Mi"sbrauch oder Manipulation der IT-Infrastruktur von einzelnen Personen.
\end{itemize}
\end{itemize}


\textbf{Strategiekontrolle}\\

Die Strategiekontrolle dient dazu die Zielerreichung zu messen und wird unter anderem durch die Zuhilfenahme von \glqq Balanced Scorecards\grqq umgesetzt. 
\\

Mit der sogenannten PDCA(Plan-Do-Check-Act)-Methode wird in quartalsweise stattfindenden Reviews die Zielerreichung gemessen. Hierbei ist es wichtig, dass nicht nur das Management eingebunden wird sondern auch die untersten Ebenen in der Organisationsstruktur des Unternehmens. 
\\

\textbf{Balanced Scorecard}
\\

\begin{tabular}{|l|p{5cm}|}
Perspektive & Frage  \\
\hline
Finanzen & Wie sehen uns unsere Shareholder \\
\hline
Markt und Kunde & Wie werden wir von den Kunden gesehen? \\
\hline
Interne Prozesse & Was m�ssen wir bieten? \\ 
\hline
Potentiale und Wachstumsperspektiven & Wie k�nnen wir unsere Potentiale noch besser nutzen? \\ 
\hline
\end{tabular}

\begin{tabular}{|p{5cm}|l|}
Perspektive & Ziele und Messung \\
\hline
Finanzen & 
 \parbox{7cm}{
wirtschaftl. Auskommen: Cash Flow \\
Gewinn: EBIT \\
}
\\
\hline
Markt und Kunde &  
 \parbox{7cm}{
 Neue Produkte: Prozentanteil der Kunden die neue Produkte nutzen\\
 Vertrauensbildung bei neuen Produkten: Wie schnell wechseln 80\% der Kunden  zu neuen Produkten \\
 bevorzugte Bank: Wie viele Kunden besitzen ein Gehaltskonto und/oder ein Aktien-Portfolio \\
}
\\
\hline
Interne Prozesse &
\parbox{7cm}{
hohe Sicherheit: Erfolg bei Penetration Tests \\
hohe Availability: �berpr�fung der Up-Time \\
hohe Reliability: Anzahl der bekannt gewordenen Fehlverhalten bzw. Bugs  \\
Neue Produkteinf"uhrung: Differenz zwischen Soll- und Ist-Zeit der Einf�hrung.
}

\\ 
\hline
Potentiale und Wachstumsperspektiven & 
\parbox{7cm}{
Innovationsf"uhrer: Anzahl der eingereichten und genutzten Patente \\
Time to market: Geschwindigkeit der Produkteinf�hrung im Vergleich zur Konkurrenz.
}
\\ 
\hline
\end{tabular}
\\

Des Weiteren wird die Zielerreichung, neben den oben genannten Kontrollinstanzen auch mit j�hrlich stattfindenden Audits vom Management gemessen bzw. "uberpr�ft und dient zur �berwachung der Einhaltung der IT-Strategie.

