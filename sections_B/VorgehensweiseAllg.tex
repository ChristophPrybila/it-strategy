Eine IT-Strategie wird in den folgenden Schritten erstellt:
\begin{enumerate}

	\item Definition von Unternehmenszielen.
	
	\item Definition von generellen Unternehmensstrategien sowie Marketing und Produktstrategien.
	
	\item Ableitung von IT-Strategie mit den folgenden Punkten.\\
	
\begin{enumerate}

		\item \textbf{Strategische Analyse}\\
Die strategische Analyse untersucht zum einen, inwieweit die IT die bestehenden Gesch"aftsziele unterst"utzen kann. Zum anderen beobachtet und bewertet sie technologische Entwicklungen und Neuerungen f"ur den Einsatz im Unternehmen.
		
		\item \textbf{Strategieauswahl}\\
Unter Ber"ucksichtigung der Ergebnisse aus der strategischen Analyse beschreibt die Strategieauswahl die Ziele der IT und setzt Schwerpunkte und Priorit"aten.
		
		\item \textbf{Strategieumsetzung}\\
In der Strategieumsetzung werden die Ma"snahmen, die Teilschritte und die zugeh"origen Aufw"ande formuliert, die zur Zielerreichung f"uhren.

		\item \textbf{Strategiekontrolle}\\
Strategiekontrolle ist ein Mittel zur Messung der Zielerreichung. Bspw. ist die IT-Scorecard ein modernes Instrument zur Steuerung der IT und Kontrolle der IT-Strategie.

\end{enumerate}
	
\end{enumerate}
